\documentclass[a4paper]{article}

\usepackage[margin=1in]{geometry} 
\usepackage[hidelinks]{hyperref}
\usepackage{tocloft}
\renewcommand{\cftsecleader}{\cftdotfill{\cftdotsep}}
\renewcommand{\baselinestretch}{1.25} 

\usepackage{minted}
\usepackage{inconsolata}

\usepackage{enumitem}

\usepackage{booktabs}% http://ctan.org/pkg/booktabs
\newcommand{\tabitem}{~~\llap{\textbullet}~~}

\usepackage{graphicx}
\graphicspath{ {./images/} }

\setcounter{tocdepth}{4}
\setcounter{secnumdepth}{3}

\usepackage{amssymb}

\usepackage [english]{babel}
\usepackage [autostyle, english = american]{csquotes}
\MakeOuterQuote{"}

\title{02.126 Southeast Asia Under Japan}
\author{Alexander Fonseca}

\date{10 Apr 2019}
\begin{document}
\maketitle
\tableofcontents
\newpage
\section{W1: Circumstances that led to Japan's going to War}
You could broadly split these reasons into \textit{a. Historical} and \textit{b. Immediate (30s - 41)}. 
\subsection{Historical}
\subsubsection{Meiji Restoration - 1868}
Japan was under a lot of pressure by Western powers. Westernization, industrialization, abolition of the Shogun and Samurai, restoration of power under the Emperor - adoption of many Western ideas.
\begin{displayquote}
	First Sino -Japanese War 1894\\
	Invasion of Taiwan\\
	Korea - a dagger pointed in the heart of Japan
\end{displayquote}
Many more militarily trained young men and demands of civilians (heavy taxation to fuel the restoration led to demands for foreign colonies) further fuelled these expansionist tendencies.\\
\\
Interestingly, it led to painters going out onto the battlefront to make paintings for...stuff?
\subsubsection{Russo-Japanese War 1905}
Rooted in Manchurian and Korean territorial issues. It was significant to the Japanese because WOOHOO Asians never die. Made the Germans semi-nervous.\\
\\
Most Japanese still preferred a pacifist stance at this time however - see the poem by Akiko Yosano - but this changes by the time Manchuria happens.\\
\\
White supremacist prodding by Wilhelm of Germany - interestingly, the Japanese looked to the Germans for guidance in the Meiji Restoration.
\subsubsection{Japan in WW1}
Beat the Germans, assisted the European allies (Britain, US, Italy and France) and profited (and took over German assets in Asia). This also earned them a permanent place on the Council of League of Nations.\\
\\
At the \textbf{Paris Peace Conference, 1919} , there was a proposal for racial equality - the Racial Equality Proposal - where there was a push for all races to be deemed officially equal. This was rejected by Britain and Australia. Japan is understandably upset.
\subsubsection{Washington Naval Conference 1921}
Naval Limitation Agreement - limiting ships in a 5:5:3 (UK:US:Japan). This snubs Japan.\\
\\
America passes the Japanese Exclusion Act in 1924, which singles out Japanese and bars them from immigrating to US.\\
\\
Then markets falter in 1926, which sparks nationalist sentiment in Japan. Well shit.\\
\\
Now the Japanese go full upsetti spaghetti.
\subsection{1930s}
\subsubsection{Manchurian 1931 - the Mukden Incident}
The "bombing" of a Japanese railway in Manchuria and the subsequent exit of the Japan from the League of Nations set the stage for shenanigans.\\
\\
And by shenanigans, I mean creating Manchukuo, using ex-Emperor Pu Yi as a figurehead head of state. The Japanese issued an ultimatum - be the head of state and have a chance of re-acquiring the title of Emperor, or be an enemy to the Japanese. Power rested with the Commander of General Affairs, and the associated assembly which was staffed by Japanese soldiers.\\
\\
After the formation of Manchukuo, Japan proceeds to migrate 1 million households to Manchukuo, which displaced hundreds of thousands of Manchurians.\\
\\
It's not a sudden annexation - the railway and all the designs for Manchuria have been there for years.
\subsubsection{The rekt of Nanking}
War against China \textbf{1937}, and it was brutal. They `accidentally' bombed the American Panay, a gunship in the Yangtze River. Plenty of killing of civilians.\\
\\
Prime Minister of Japan (also Prince), Fumimaro Konoe escalated the war in China, and pushed for the Tripartite Act (signed Sept 1940) which created the Axis.
\subsubsection{Pearl Harbor 1941}
2 hours after the start of the invasion of Malaya, they dropped by the American naval base Pearl Harbour and by that I mean they dropped bombs on Pearl Harbor. Chiang Kai-Shek and Winston Churchill were pretty hyped.
\subsection{How Social Awkwardness drove Japan to War}
Emperor - scared of upsetting the military and govt because of previous attempted coup\\
\\
Konoe - PM of Civilian Govt - fuffed about with working out a treaty/agreement with the Americans.\\
\\
Mitsumoe - Foreign Minister of Japan - Sees allies in Germany, Italy, Russia (because of German-Russian non-aggression pact),\\
\\
Military Arm - two factions: Army and Navy. Army and Navy commanders also trying to impress their jingoistic subordinates - all trying to impress each other. They're all secretly hoping that the Civilian Govt is able to work out a diplomatic solution with the Americans so they don't have to fight.\\
\\
\textbf{Homework}\\
Similarities between the readings/comparisons and what not.
\section{W2: Before 1941 - Part 2}
The crux of the Japanese-American conflict was China.\\
\\
Why...why did Germany and Japan even form the Tripartite Pact? They're not exactly on good terms, given that the Germans had heavily invested in Manchuria and China, and worked with Chiang Kai-shek. Japanese students were also heavily harassed in the 1930s due to rising fascism and racism. Japan also distanced itself from Hitler's position on the Jews because the Jews were heavy investors in Japan.\\
\\
Probably to keep out the Americans - to have enough `threat value' to deter the Americans.\\
\\
Anyway, ye olde Americanos decided to cut Japanese oil supplies when the Imperial Japanese Army invaded French Indochina to blockade Chinese oil supplies.
\subsection{Diplomacy}
The Draft Understanding bounced between Japan and America, which outlined an understanding between US and Japan. US wanted Japan to break out of the Tripartite pact because Germany wasn't being a nice kid (1941), to get out of China, and remove Japanese forces from Indochina (remember the blockade on Chinese oil?). Japan wanted them to stay neutral (a neutrality pact).\\
\\
\textit{Hypothetically} the Japanese would have benefited heavily from breaking the Pact when Germany enacted Operation Barbarossa - where the Germans broke the neutrality pact between the Russians and Germans. Matsuko the dipshit proceeds to lobby for attacking the Russians as a sort of show of solidarity - the Emperor realises he's a dipshit.\\
\\
Ye boi Kafu realises the Japanese govt. is going bonkers, and he, in his diary, wished that the Americans would come over and smack the shit out of Japan so that they'd have sense.\\
\\
Meanwhile, the Japanese are getting hit by wartime poverty, but the Japanese, inflamed by wartime passions and nationalism and what not, still supported the war in Chinese. The \textit{Total War Institute}, in their simulations, realise that the war isn't going to work out well - it'll start great but get rekt afterwards - but ye boi Tojo says that the Japanese spirit, being an unknown factor, renders the simulations meh.
\subsection{Konoe is a dipshit}
Konoe's ideal diplomatic plan with the American is to keep the jingoistic shows, and when they finally meet up with Roosevelt, they'll somehow work out some face-saving solution. The Emperor will berate the Army and that'll put them in their place, which in turn gives way for a brokered peace.\\
\\
In actuality, the Americans didn't quite trust Konoe - him appearing when Japan left the League of Nations and signing the Tripartite pact gave him some pretty bad street cred. Also, because he couldn't come up with his own terms for the proposal to America, he ended up sending the Army terms (which were harsher and less pleasing).\\
\\
Konoe is also a massive dipshit. He ends up setting up an ultimatum as to when to attack the American (end October) should peace not work out. The Emperor calls him out on his bullshit, and Konoe still refuses to withdraw his words.\\
\\
So early October rolls in - Konoe insists on not withdrawing from China, because he feels that once the European theatre finishes up, Japan is next. He insists on staying in Manchuria and China because of the blood and money spent on the Chinese theatre - what a waste it would be to pull out now. (The IJA generals in Manchuria want to GTFO because the loses aren't worthwhile.) He consults fellow dipshit Tojo, asking if war could be avoided, and words obfuscated by diplomatic phrasing, but ye boi Tojo is dumb and insists.\\
\\
Meanwhile, Yamamoto, bossman of the Navy, keeps sharing in private that the Navy isn't strong enough to assure a victory - rather he thinks Japan will lose. However, he can't publicly say that because \textbf{FACE}\\
\\
Konoe steps down in mid-October. Tojo steps up.
\subsection{Hit me baby one more time [Cohort 14/2]}
Japan wanted to break out of the ABCD encirclement: \textit{Americans, British, China and Dutch East Indies}. Many efforts were made, not just by the Japanese but also the Thais (who wanted to reclaim some of the taken French Indo-China region) to acquire intelligence. To do this, they did a lot of `cultural espionage'. Japanese goods and products were incredibly popular at this point, which meant Japanese businesses could act as sources of intelligence - in fact, there were plenty of spies that took on such roles, pretending to be businessmen.\\ 
\\
The Japanese also made it a point to purchase lands and set up shop in proximity to strategic locations - that made for some fun intel collection times.\\
\\
Oh, it also happens that the Japanese bicycles were \textit{really} popular in Malaya. \textit{Hmm}.\\
\\
The Japanese were also actively recruiting local supporters - Suzuki Keiji was a Japanese general who, based in Thailand, tried to spread the idea of the Greater Eastern Co-Prosperity Sphere.
\subsubsection{Ye Boi Aung San}
Father of Aung San Su Kyi lmao.\\
\\
Led the \textit{Thakins}, an anti-Imperialist movement comprising of university students and learned men that resented British rule and that the current leaders were stooges. They're also pretty upsetti that the Brits lumped them with the Indians. As the Brits and Burmese leaders tried to crack them down, they turned to the Japanese for help - a sort of deal with the devil, because the Japanese were imperialist too.  This gave rise to the Burma Independence Army - starting with \textit{The Thirty Comrades} and Keiji Suzuki. The Japanese train them, supply them, provide air cover and such, and the Burmese were so incredibly thrilled by the idea of Burmese independence and the Burma Independence Army swelled with Burmese support.
\subsubsection{India}
Similarly, the Japanese do something similar with the Indians. Rashbehari Bose does something similar in the1920s, marries a Japanese wife to get protection from the Japanese, and heads to Tokyo and sets up a curry store. \\
\\
Heck yes Japanese curry.\\ 
\\
When the Japanese want to encourage the Indians to stir shit, they can't call upon Rashbehari Bose because he's old. Subhas Bose comes into play - very intelligent civil service dude who really didn't like the British, and in fact he joined the Indian Congress Party, where even Gandhi recognized the value of people like him. Meanwhile the Congress was split into two sides, one pacifist and the other war-hungry.\\
\\
So Bose pulls some shit to gather support. He stirs controversy by demanding the renaming of a monument - this monument is the Howell Monument, where an Indian king killed a shit ton of British. He turns it around talk about the Indians killed by the British colonial powers, which triggers the Brits and ends up with him getting jailed. He gets a massive amount of support, which gets even bigger when he goes on hunger strike. This leads to him being under house arrest, and he escapes. Some shit later and...tbh idk. We'll cover his story later.
\subsubsection{Tan Kah Kee}
A billionaire in Singapore, he organised the local Chinese and rallies them to fund the Chinese resistance. He goes to meet Chiang Kai-shek and doesn't like what he sees - he's extravagant, and the culture of the people under Chiang wasn't exactly ideal. He later goes on and meets Mao Zedong, and likes what he sees - a polite, reasonable looking dude (haha). As he returns to Singapore and vouches for the Communists, this eventually encourages a truce between Chiang and Mao.\\
\\
Then the Brits feel kinda worried, because the Malayan Chinese are getting rather powerful. As usual, they get arrested and thrown into jail, which drives the Chinese underground. Nice work dipshit.
\subsubsection{The Malays}
The Japanese wanted some local support to mess the British up - a fifth column to support them when they roll about. Who better than the native Malays to do so?\\
\\
Anyway, the Japanese push the story of the Tiger of Malaya, a Japanese man who allies with the Malays and shrekts the British, giving away his spoils to the local poor. He's a real dude but the Japanese hype up the story to ridiculous levels - that way they seek to build solidarity between the Japanese and Malays.\\
\\
Simultaneously, there was an uptick in Malay newspapers, coinciding with a burst of young, educated Malay leaders. This acts as a medium that inspires the Malays.
\subsubsection{Dem brits luv their empire}
\begin{displayquote}
	Malaya - the Industrial Diamond of the British Empire, the largest exporter to US after Canada, and producing half the world's tin\\
	\\
	India - The Jewel in the Crown, because it big.
\end{displayquote}
The Brits love their Empire. Makes money - in fact, there's a lot of money coming from opium taxation. The Brits encouraged opium smoking for the Chinese, and alcohol for the Indians. Remember that these come from the poorer sections of society.\\
\\
On the other hand, while the War was playing out in Europe, the Brits were really chill here, being removed from the violence, and were generally unprepared for the incoming storm (\textit{of Japanese}).\\
\\
The disparity in quality of life in this part of the world is..stunning. The Brits placed themselves above, while the poorer Indians, Chinese and Malays worked in the plantations, manual labour and agriculture. They placated these groups with alcohol, opium and religious and cultural concessions - the British tried to make friends with the Malay rulers.\\
\\
There's also this competing theory that the Malays weren't as poor as depicted, because there's a non-negligible slice of Malays running businesses and doing trade.\\
\\
The Japanese come along and nudge the Malays to reclaim the land for themselves. "Kick out the Brits," they said, "and make this land your own. A nation for Malays." They capitalised on the British compartmentalisation of society to inspire disobedience.
\section{W3: Pan-Asianism}
\begin{displayquote}
	We're doing some of last week's material because we didn't cover it.
\end{displayquote}
The idea that there's some unifying identity for Asian nations that would allow Asia to resist the West.\\
\\
The Western imperialist powers declined to give Asian people the same things that they gave their own citizens - self-determination, clear borders, votes and what not. They viewed Asia as the "waiting room of history" - where the people were not ready to create their own civilization, and needed Western guidance to achieve their potential - and hence justified Western presence.\\
\\
\textbf{Pan-Asianism} isn't just a Western concept - there exists commonalities between the Asian countries. Culturally, Confucianism, Buddhism and Hinduism are common threads that bound the Asian countries together.\\
\\
Japan identified itself as Asian, but also as a cut above the rest - some mental gymnastics took place when they tried to justify their shenanigans in Korea and China, and that they aren't a colonial power themselves. 
\subsection{Teaist Pan-Asianism}
By Okakura: A pacifist idea of Pan-Asianism, rooted in cultural and spirituality and tea. Asian `moral' soft power rooted in art, culture and philosophy, in contrast with the Western focus in science and politics.
\subsection{Sinic Pan-Asianism}
China needs help - too many problems and too varied a people. A weak China made for a poor Russian counterweight, and having Japan come along and fix China up would do Asia good. Japanese, Korean and even Chinese investors poured funds into this - until Japan started screwing around with the Chinese.
\subsection{Meishuron Pan-Asianism}
The Empire of the Rising Sun needed to be protected - a throwback to the samurai era of honor and what not. Somehow  a lot of Japanese threw their lot with this. Japan needed to be militarily strong to protect the Emperor, and by extension, Asia. Under the Japanese umbrella would be the Greater Eastern Co-Prosperity Sphere.\\
\\
Yes, the Japanese believed in these idea, but we can't ignore the practical aspects of this war. They did need the resources.
\section{W3: Japan in Malaya}
\begin{displayquote}
	Thomas Shenton - Governor of Singapore\\
	\\
	Percival - British Commander of Malaya and Singapore\\
	\\
	Yamashita - The Tiger of Malaya, Commander of the Japanese forces, and also a political enemy of Tojo. Still, a capable man nonetheless.
\end{displayquote}
The Singapore strategy - start with a naval base, some cannons, and get reinforcements from India and Australia. If shit goes down, get the Americans (didn't work because of \textit{Pearl Harbor}) or reinforcement from the Middle East (Churchill's shifting theatres of war).\\ 
\\
The English had this incredible notion that the Japanese were grossly incapable of fighting - even when clearly outmatched. With the Japanese rolling in with 600 planes, mostly modern planes like the legendary Mitsubishi Zeros, the British were still fuffing about with 90 old planes like Buffalo Brewsters. You can see how this wouldn't work.\\
\\
The Thai were uncomfortable because to the north was Burma (under Brits) and to the south was Malaya (under Brits too) and thought the Brits had designs on the isthmus of Kra.\\
\\
The British planned Operation Matador as a countermove to Japanese landings in Malaya, but that would piss of the Thai.\\
\\
The advance of the Japanese was incredibly rapid - not just because they were properly equipped for the jungle, but also because the Brits very easily retreated. Contingents of Indian troops would surrender and defect because the Brits just gave in and ran.\\
\\
Disappointing.\\
\\
So repeatedly, they'd cede more land and say `let's protect the further south'. You can see how this doesn't work.
\section{W4: A surprise presentation by Roland! :D}
After WW2, France made efforts to reconnect with French Indochina.\\
\\
Post-2nd Opium War, there were many concession regions which were ruled by colonial powers (French, UK, Germany and Japanese) which were not considered Chinese territory. As the Japanese rolled down from Manchuria, they went through these areas to reach Tianjin.\\ 
\\
For the French, they set up clear markers of identity - flags and what not - to identify themselves so the Japanese would know not to attack them. These international territories were safe from Japanese shenanigans, which meant the Chinese would try to slip into said territory for safety.\\
\\
Some modicum of collateral damage though.\\
\\
The Japanese rapidly seized existing (remaining) infrastructure because they wanted to leverage on that to springboard their forces.\\
\\
Flood during 1939, during Japanese occupation of Tianjin.\\ 
\\
"Shanghai was historically built by foreign powers." - you can see the grid design characteristic of Western planning.\\
\\
The marks of French influence is still there however - a lot of colonial era buildings, and French markers of ownership - within the French Concession area of Tianjin.\\
\\
Interestingly the Japanese seemed ready to take over the entirety of the area - as a courtesy call, a Japanese officer visited the French consulate, and when meeting the French ambassador, the officer placed his feet atop the table. It outlines what the Japanese grand plan was - usurpation and rule.
\section{W4: Back to our scheduled programming!}
While the Japanese advance on Malaya may appear to be a swift, short campaign, it is also true that the Japanese did take some serious losses against the British. The KMM (a Malay unit allied with the Japanese) gave intelligence to the Japanese, and at the same time there were also non-Japanese groups that stuck around and helped the Japanese calm down the populace.\\
\\
Mustafa Hussein (of the KMM - Young Malay Nationalist Organisation) soon realizes the Japanese may not be so benevolent after all - because the Japanese seemed to look down on the Malays, and thought of them as children.\\
\\
That being said, this isn't the sole opinion on the Malays. There are those who stood against the Japanese.\\
\\
Matador didn't work because no one was willing to declare war on the Japanese and Thai. The Thai were on the Japanese side - further reinforced by the gift of 4 states to the Thai - and already had shown their friendliness.\\
\\
Why were the Japanese so quick in advancing? (1) British roads made movement easy and (2) the Japanese had planted supplies and requisitioned resources from Japanese businesses along the way. Especially the bicycles.\\
\\
The British knew they fucked up when the Japanese rolled along, found the Prince of Wales and Repulse, and bombed them. The Brits had no air cover available, which made them even more vulnerable.\\
\\
As the British retreated, they employed scorched earth to delay the Japanese - they burned everything from oil, food, vehicles and even money (that's when that Tan Kah Kee realised the Brits weren't really interested in saving Malaya). Penang was bad in particular because the British just bolted on the first sight of danger - in fact, even the Union Jack was lowered by an Indian newspaper editor. The Japanese flexed on the British real hard on that. The British flight meant resources came down to Singapore, because they thought Singapore was impenetrable.\\
\\
The Japanese moved into the Sultan of Johore's buildings (he insisted the British not touch them) and set up military intelligence centers in there, and the Brits blew up the causeway.\\
\\
Here we go.\\
\\
Rapid Kota Bahru assault, followed by projection of air power. Heavy bombing.\\
\\
"No more law and order - then the looting started."\\
\\
The British call for an evacuation for all the European subjects - leaving behind their Asian subjects. This is some serious betrayal.\\
\\
Yamashita's bois roll down Malaya with tank, trucks and bicycles - the "Driving Charge". These soldiers are battle-hardened veteran who have fought against Chiang Kai-shek in China. The IJA inspire serious terror - especially the Chinese - some run into the rubber plantations and forests. The Japanese soldiers strip down people for clothes and watches (anything expensive, really).\\
\\
They also start their propaganda campaign for the co-prosperity sphere.\\
\\
The Japanese are not interested in nation building, and speak of favouring malays under the banner of the rising sun.\\
\\
Some KMM units augment the Japanese forces advancing south.\\
\\
Differential treatment by racial lines - the Chinese are heavily targeted by the Kempeitai as part of the Sook Ching because of their sympathies for Chiang Kai-shek. In particular, they want to eliminate the MPAJA - an anti-Japanese army - comprising largely of Malayan communists.The British release the communists in Singapore to fight their common foe.\\
\\
"Now, by order of his Imperial Majesty, kill them." - the absolutely authority of the God-like Emperor. The Officers borrowed that power to compel the Japanese soldiers to kill the Chinese. Amidst those buried were those who were still alive.\\
\\
With the Chinese numbers thinned, the Japanese turn their attention to molding Malaya - draconian laws - where beheadings were used as punishment for many things. Sentries are set up everywhere, with soldiers being vested with a fragment of the Emperor's authority, and hence require that all people bow when passing.\\
\\
The Japanese spirit `Seishin' - being on time, punctual and strong. Time was set to Tokyo time, and calendars switch to the Japanese one. The Japanese language is instituted, and kids have to sing patriotic songs. Food is used an incentive - 2 tins of rice and a packet of sugar - given to the students.\\
\\
By 13, you'd be required to work - and you'd have to co-operate with the Japanese.\\
\\
The Japanese of course took comfort women - sex slaves - and service easily 60 IJA soldiers a day. \\
\\
British India's next on the list - and they want to use rail to ship their troops over to the Indian border. The Japanese need to link up the railways from Malaya to Thailand - and so begins the building of the Death Railway. They need to build this railway through mountains and jungle fast. People are tricked into working on this - they are told that the pay is good and food is plentiful. Reality is often disappointing - food is scarce, and the Japanese are cruel and abusive. 13000 Allied soldiers and 70000 civilians perish over the construction period.\\
\\
Food was scarce. Very scarce.\\
\\
The MPAJA is fairing badly - lacking training, food and leadership - and rely on food from Chinese villages - problem: the Kempetai have spies out there, and those with links to the MPAJA get tortured - their favourite tool is waterboarding, referred to as Tokyo Wine.\\
\\
Where the Indian and Chinese are tormented, the Malays were clearly favoured, giving them food and jobs. These does not end well. In the village of Sungei Lui, a Malay guard posted by the Japanese is dragged away and slain by the MPAJA - who seek to acquire food from the village. The Japanese, infuriated, descend upon the village, separate the Chinese and Malays, and slay the Chinese. Malays could only watch.\\
\\
The British Force 136 steps up their efforts to assist the insurgent elements against the Japanese. Meanwhile, KMM under Yacob Ibrahim still tried to squeeze the Japanese for independence. They assemble August 7, 1945 and try to declare independence - but surprise! The Americans drops a nuke on Hiroshima on August 6. Japan surrendered quickly. \\
\\
In the power vacuum, with the British still on the way, the MPAJA proceeded to claim victory. They proceeded to target those who were perceived to have joined the Japanese. These were "People's Courts" but truly they were Kangaroo courts. \\
\\
The British finally came back and set up the BMA - \textit{Bring Misery Again} the people call them - but it seems that there isn't much interest nor support for this.
\section{W5: Japan and Malaya}
The Japanese had seeded and fanned independence movements within the Malays, given their funding and support from the KMM. They also leveraged on Malay Friday prayers to spread paraphernalia and pro-Japanese messages - which imbued a sort of strength into the clergy class.\\
\\
However, there was some low-level resistance against the Japanese (clearly). Breakoffs from the MPAJA would raid Japanese supply trucks, and the supplies stolen bled into the black market.\\
\\
The Japanese arrival catalysed the formation of the idea of a Malay nation. Apart from providing support to the KMM, which already wanted an independent Malay nation, their rule also indirectly gave rise to several Malay newspapers - this also caused a shift in linguistics, as Roman letters were adopted as the standard script for Melayu, and gave rise to some new phrases (in fact, when the British finally return, there was a whole new vernacular they'd have to pick up).\\
\\
The KMM eventually got miffed that the Japanese weren't actually keen on a Malay nation, and kept pushing for it. They eventually met and worked together with the Communists, which is weird because the Communists are allied to the British, and the KMM was allied with the Japanese. Eventually, the Japanese found the KMM more trouble than they were worth, and banned the KMM, with adviser roles being given out to the leaders of KMM.\\
\\
Then, in 1943, Burma gets a sort of independence with their own government (which is a puppet government, of course). This excites the Malays and gives them hope.\\
\\
Then \textbf{SYKE} the northernmost 4 states are given to Thailand, which to the Malays was an incredible betrayal. Mustafa Hussein said "\textbf{NOPE} screw dis I'm going back to perak bye you little shits".\\
\\
Japan then offers military training and the militarisation of the Malays, creating their own army - with 2000 strong, Japanese uniforms, training and equipment, led by Ibrahim Yaacob. This army was set against the MPAJA, but that really didn't work out because 1) these people aren't really warriors, and 2) many of these people thought that the MPAJA weren't that shitty and it was wrong to fight their own people.\\
\\
Food supplies were shitacular because the 4 northern states were the rice-producing states of Malaysia. (Meanwhile in the 4 states, while they were nominally Thai, but there's weak policing by the Thai and they're generally run by the Malays). Without good food (since most of the food was channelled to the Japanese war effort), there was a shift toward use of tapioca. It's calorie rich, but not nutritious, which led to a lot of growth stunting.\\
\\
THEN YAMASHITA MY BOI SEZ

\begin{displayquote}
	"Yo chinese people fuk u too gimme 50 million dollars"
\end{displayquote}

\noindent And the Chinese somehow scraped together 50 million dollars (by borrowing from a Japanese bank).\\
\\
There's a slow breakdown of the notion of the plural society because of the common denominator of \textit{desperately trying to survive} lubricated social exchange when people met in the markets (and black markets). Old British systems and ideas like a segregated plural society, and business/logistical systems were thrown out of the window. People improvised, and those improvised systems stayed.\\
\\
KRIS - kesatuan rakyat indonesia semenanjung. A greater Malay nation including Indonesia. A notion that bubbled up at the end of the War.
\section{W8: Burma}
After the initial gains made by the Japanese, they got a bit excited and decided "Hey, let's take Burma too. Can't be too hard.". They were right.\\
\\
Dec 1941 - the Blitz of Rangoon. Fierce bombing that killed 2000 Indian labourers. Burmese fled the towns, Europeans and Anglo-Burmese were evacuated.\\ 
\\
The British decided early on that they would sacrifice most of Burma for it - they just wanted to hold the north, as that's where India was connected, and where the Yunnan Burma Road.\\
\\
Chinese Army under Stilwell enter to support the Allies and defend Burma Road (Stilwell disliked Chiang Kai Shek, thinking him a two-faced liar, and had interests in North Burma).\\
\\
Still, no British troops deployed by \\
\\
Chennault - an avid supporter of Chiang Kai Shek, with the American Volunteer Group "Flying Tigers". He had worked under Mdm Chiang Kai Shek, and had gone over to train and support the Chinese. The Flying Tigers set up at Kunming, which placed them within 50 miles of Hanoi - within fighter striking range. Within the day of the Tigers arriving, the Japanese sent out a formation.\\
\\
America was heavily invested in keeping China free.\\
\\
Appalling lack of sanitation in Rangoon - `Life begins with the sweeper' - disease ridden.\\
\\
Rampant looting because of the Japanese proximity.\\
\\
Dorman-Smith, `the most unpopular man in India' - in charge of Rangoon, but based in India. Did not execute scorched earth well - pre-emptively destroyed boats on the east of India, which led to starvation due to a loss of livelihood.\\
\\
Reinforcing Singapore meant no troops available for troops, and the Australians refused to send in troops from the Middle East.\\ 
\\
While the British people fled, the Anglo-Burmese and Anglo-Indians stayed. It's Malaya all over again.\\
\\
600,000 people flee from Burma to India. 80,000 perish.\\
\\
Burmese fled to the interior, Chinese return to Yunnan, but the Indians...they had to head through jungles along the Chindwin River.\\
\\
One small perk that the Indians had was that the Japanese were really not interested in killing Indians - Chinese and Caucasians, they'd kill on sight, but the Indians were slated to be allies.\\
\\
Initial attempts to flee were mostly by ship, but ships made for easy targets for bombing.\\
\\
Racial hierarchy would come into play here. British racism would be on full display as they made all excuses possible to secure their escape.\\
\\
The Thakins (Burmese Independence Army), with 12-18k strength.\\
\\
The IJA, with 300k strength waddling in.\\
\\
When the Thakins were released with the Japanese advance, the Thakins started a murder spree of Indians. The IJA's arrival stopped that, because the Japanese saw the Indians as possible allies.\\
\\
The BIA and local Burmese gangs went on a rampage, inflicting their own form of `justice' on people, and also targeted minority ethnic groups.\\
\\
Shan: helped the Japanese and the Thakins
Karen: targeted and killed by the BIA due to divided loyalties.
Chins: helped the British
Kachins: helped refugees
naga: helped refugees
Lushai: helped British (Lushai Nationalism)
Muslims: the Rohingya\\
\\
A rivalry sprouts between the BIA and IJA as they try to claim and acquire support from the villages.\\
\\
Monks fundraising for the BIA\\
\\
\textbf{Significance of 1942 on Burmese society}\\
Rural society + priests + army came together to form Burmese society - this sets the stage for post-war Burmese society, where the priesthood have an inordinate amount of power.\\
\\
Fall of Mandalay\\
\\
Bombing of Hill Station of Maymyo\\
\\
Myitkyina - the last airport to be lost to Japan. An effort would be made to get it back - because it's part of the Ledo boat.\\
\\
The Japanese eventually decide that the BIA was a pain, and did not want the BIA in office. There was no one to prop up - the British had dismantled and dispersed the Burmese royalty long ago. They installed Ba Maw, a senior member of the BIA who was more benign.\\
\\
Tea plantations and plantation workers saved the day. Elephants helped alleviate a lot of problems - they're great at jungle terrain, and helped make roads and move goods.\\
\\
Blatant racism of the British - the dual roads. Cholera was rife. The local tribes came along and helped.\\
\\
The Ledo Road connected India to the Burma Road, to ensure that China would retain supplies. Americans and Brits were really really intent on keeping China, and Burma was an proximate beneficiary.\\
\\
Loss of Malaya = no quinine available to treat Malaria. Oh dear.\\
\\
Burning of boats (Smith's scorched earth) and the Rangoon bombing caused prices in Calcutta to skyrocket. Famine in East India because of 1) scorched earth, 2) bad weather conditions, 3) centralization and stockpiling of resources by Britain, 4) resources directed to military. 1 million people because of that.\\
\\
\textbf{Tribal/Highland participation}\\
\\
Zomia - the highland region of south east asia that were generally outside of the rule of governments. \\
\\
Ursula Graham Bower - anthropologist who hung out with the Nagas of Northwest Burma. Eventually, she is called upon by the British govt to mobilize the Nagas to help. They do a bit of guerilla warfare, and help with the evacuation.\\
\\
After the strong initial push into Burma, British and American forces push hard to liberate the Northwestern area of Burma - America was probably trying to help China, because that opens up the Ledo road.\\
\\
Japan realises they're not doing so hot - so they grant Burma independence.
\section{W9: Indonesia under Japan}
We're reading a diary! A very different kind of primary source, probably quite content-dense. There might be terms that we don't know - political terms in particular - so just google.\\
\\
The Japanese invasion of Indonesia was rapid - within a month, they took the country. The Indonesians welcomed the Japanese invasion - so there was less killing - and while there was resistance, it was relatively little.\\
\\
The Indonesians were fascinated by 1905 - the defeat of Russia by Japanese hands.\\
\\
The Dutch took on the Ethical Policy in 1901 to late 20s - the most ethical colonial rule amongst the colonial powers - and 1916 Peoples Council (Volkgraad) gave some limited governance to the Indonesians.\\
\\
In 1940, 0.4\% of the Indonesians were Dutch - that's the highest. By that time, Indonesia was 70 million strong.\\
\\
The Japanese rule was `prophesied':
\begin{displayquote}
	"The Javanese would be ruled by whites for 3 centuries and by yellow dwarfs for the lifespan of a maize plant prior to the return of the Ratu Adil: whose name must contain at least one syllable of the Javanese Noto"
\end{displayquote}
The Chinese, of course, were worried. Oh boy.\\
\\
Same strategy employed by the Japanese - spy system and a large corporate presence, and Indonesian nationalists were invited to Japan (esp Muslim parties).\\
\\
Sukarno, the future president:
\begin{displayquote}
	"Yes, Independent Indonesia can only be achieved with Dai Nippon...For the first time in all my life, I saw myself in the mirror of Asia."
\end{displayquote}
As the Japanese entered, the Indonesians attacked Dutch civilians and military - Muslim leaders led attacks against Europeans, Chinese, Javanese Christians. The Japanese had given the Islamic leaders the opportunity that the Dutch denied - an Islamic nation.\\
\\
Some pressed the Japanese for political leadership - Hatta and Sukarno - while others stood against the Japanese (Amir Sjarifuddin and Sjahrir). The KNIL (Dutch army) were sent to camps, while another 100 000 Europeans and some Chinese were interned.\\
\\
Japan was concerned about an Australian counter-attack, so they worked with the Indonesians to set up a security force.\\
\\
For administration, Japan split Indonesia into 3 sections - Sumatra was included with Malaya, which fuels hope for Malaya merging into the greater Indonesia.\\
\\
Sukarno - a charismatic, powerful, and also rather vain person. The Japanese saw him as an incredible force to move the people, so they set him upon the Indonesians to convince them to work for the Japanese - the Romusha.
\begin{displayquote}
	"Let's iron America and bludgeon the British!"
\end{displayquote}
Several millions were moved to Java to work, 2-300 000 sent abroad (including the Death Railway). Very high mortality.\\
\\
Conference for Islamic leaders in Jakarta in 1942. Japanese decided that urban Islamism wasn't useful. They looked to rural islam - NU, Muhammadiyah -which had schools, welfare groups and many informal ties to the interior. The kiyakis refused to bow - literally, to the Emperor.\\
\\
Sukarno was flown to Tokyo in Nov 1943 to be decorated by the Emperor, but Tojo refused use of flag or anthem.\\
\\
In Oct 43, Pembela Tanah Air (PETA) was set up, a volunteer Auxillary guerrlla force, which drew 60 000 recruits in total.\\
\\
On the village level, \textit{Rukun Tentannga (Tonari Gumi)} were set up - cells of 10 - 20 families for information and indoctrination. This form of village level policing helped the Japanese keep track of things.\\
\\
This sort of organization, which the Indonesians never did have under the Dutch, allows the Indonesians to wreck the Dutch when they return. The Japanese left 1945, and the Dutch were kicked out 1949.\\
\\
Sukarno galvanized the youth - Pusat Tenaga Rakyat (Putera) - and they became incredibly powerful. So much so that Sukarno felt threatened by them - they squeezed him to declare independence.\\
\\
1945 Feb 14 - PETA uprising in Blitar: The Indonesians now have some sort of military capacity.\\
\\
1945 July - the Japanese tried to unite the yout movement and the army in a combined peoples movement. It failed as the youth had become to politicized and angry (recall: Rokmusha, starvation)
\begin{displayquote}
	Famine in Java 44-45: 2.4 million die. Overall, 4 million might have died during 41-45 (5\%). 30 000 Europeans (30\%) too.\\
	\\
	More than 5\% of the population. Malnutrition (the Japanese co-opted all the food for the armies), and literally worked to death.
\end{displayquote}
Sukarno asks for a \textbf{religion free nationalism}. What?!\\
\\
Atom bombs dropped, Japan surrendered, the Putera grab Sukarno before he heads to Rangoon to negotiate with Malaysia, and force him to declare independence on Aug 17.\\
\textbf{ABC video, Sukarno and the Japanese} - that Sukarno collaborated with the Japanese as a big-picture thing, to enable the Indonesians, and that Indonesia declared independence prematurely.\\
\\
The Japanese surrender was sudden, and the Japanese just let go of Indonesia. It took some time for the Dutch to return (see: infighting between Dutch and Allies), so for a period of time, Indonesia was... up for grabs? A political vacuum was in place, but the Japanese were still around and in-charge. The Japanese still armed in some places - they would protect British troops, sometimes even fighting for the British.\\
\\
MacArthur (American) did not invade to help the Dutch - a lot of bloodshed.\\
\\
Many Japanese joined the Indonesians to fight against the Dutch - they became national heroes (see Abdul Rachman Ichiki Tatsuo)
\section{W10: The PETA Revolt in Indonesia}
Few written sources - largely survivor testimonies.\\
\\
Formed October 1943 as \textit{Tentara Sukarela Pembela Tanah Air} - Voluntary Army for the Defence of Land and Seas. Strikingly, it was \textbf{volunteer} basis, and was largely from \textbf{upper class families}. Many signed up - at the time, the Japanese were still viewed \textbf{positively}.\\
\\
Creation of independent, non-hierarchical \textit{daidans} (battalions) - not hierarchical but independent, and intra-\textit{daidan} contact was forbidden. This was to prevent the possibility of revolt.\\
\\
Meant to serve as a standing force against possible reprisal from the Australians, or attacks on Indonesia.\\
\\
Indonesia had long been `demilitarized' under Dutch rule, and the Japanese militarization of Indonesia empowered the Indonesian people by equipping them with the necessary skills and equipment.\\
\\
Each \textit{kabupaten} raised a \textit{daidan}. They were pulled out from their civilian lives and put through rigorous military training. When they returned after training, they found that civilian life had changed for the worse - farmers were forced to sell rice, eggs were bought by the Japanese at cute-rate prices ostensibly for PETA, but it ended up for the Japanese.\\
\\
Scrap iron was being taken, with ornaments and young women forced to go to `Tokyo' for `study'. \\
\\
PETA officers would be slapped by the Japanese, and that really ticked off the Indonesians because the Japanese claimed that they were equals.\\
\\
Many taken away to be \textit{romusha}, forced labour for mines and the Death Railway. Blitar \textit{daidan} got to see some of this, but these \textit{romusha} were slightly better off.\\
\\
Indonesia was host to a lot of Dutch, and let's not forget that they, the colonizers, ended up as victims as well.\\
\\
Japanese used other PETA troops from Blitar \textit{daidan} and \textit{Heiho} to \textbf{suppress} the \textbf{revolt}, which made it difficult for the Indonesians to fight each other (they're all nationalists after all). The rebels withdrew into the mountains, which helped them endure a bit longer. Eventually the revolting troops surrendered with conditions. The Japanese granted relatively \textit{lenient} sentences because they had to handle the Indonesians carefully - if they executed all the troops, that would inspire rebellion in the rest.\\
\\
\textbf{Supriyadi} - leader of the PETA rebellion. Disappeared after the forces withdrew into the mountain. He was slated to be the minister of defence, apparently, but he just...disappeared.\\
\\
Blitar's pretty popular, because it's a story that advances the cause of nationalism or valour (it's rather romanticized). An old man eventually claimed to be Supriyadi, and there was a lot of interest because of the tale - and he made many plausible claims. The one \textit{startling} one is that he was present at the declaration of Independence, and was simply not noticed because he wasn't dressed formally, and kept a low profile.\\
\\
Whether it's true or not is irrelevant - it's the fervour that it inspired that we should note.\\
\\
There were also \textbf{Japanese} who devoted themselves to the independence of Indonesia against the allies. They gave up their weapons to the Indonesians to arm them against the Allies - cheeky start, really.
\section{W11: The Philippines}
1898 Battle of Manila Bay - the Spanish surrendered to the Americans, and gave up the Philippines. The reason for the fight was \textit{Cuba} rather than the Philippines itself (Spanish American War 1898) resulting in the Treaty of Paris.\\
\\
America pays Spain \$20 million for infrastructure (and ownership) of the Philippines.\\
\\
Nationalists say \textit{we want our country}, not a \textit{change of colonial power}. The Americans were never quite the colonial power compared to the Spaniards, Brits or the Dutch, but for the Filipinos, the Americans were just another white power coming over.\\
\\
The nationalists fought against the Americans for 3 years, and unsuccessfully so - they surrender in 1901. America was brutal in their clamping down against the revolutionaries (waterboarding, and widespread killing). In 1935, the Commonwealth of the Philippines was set up, because the Americans insist they aren't a colonial power.\\
\\
Then the Japanese attack.\\
\\
Most of the fighting is centered on Luzon, with two other strikes on the East and South. The Americans floundered, the same way the British did - the Japanese attack was fast, precise and brutal. To the Americans' credit, they held on longer because\\
\\
The Americans sought to preserve Manila (because they put a lot of resources into Manila) and decide not to fight in it - they make it an `Open City', where they accept that the Japanese can take over it, contingent on not-destroying-the-whole-place.

\begin{displayquote}
	It didn't work, because the Japanese still opened their assault with aerial bombing to soften the (non-existent) defence.
\end{displayquote}

\noindent The American and Philippine armies move to Corregidor, a small island, to hold their ground. Fighting is brutal - 5 months from December 30th till surrender.\\
\\
MacArthur is recalled to Australia by America. Surrender on 6th May, under Wainwright.\\
\\
The \textit{Bataan March} - 80,000 POWs marched to camp \textit{O'Donnell}, and only 50,000 reached the camp. Horrific conditions, with the Japanese just pulling random POWs into the bushes to be bayoneted so the Japanese troops could steal their watches. 6 days without food to walk 60 miles - those who fell behind where killed.\\ 
\\
\textit{Hell Ships} - Japanese ships that transported the POWs to labour camps. One. carrying POWs, was sunk with their cargo on board - out of 662, only 82 survived.\\
\\
The Japanese set up a government in the Philippines - not very popular - and a government-in-exile headed by Quezon (President of the Philippines Commonwealth) in America. \\
\\
Despite Japanese efforts to win over the Filipino populace, Filipino sentiment was still strongly anti-Japanese.\\
\\
The Battle of Midway represent a sort of turning point in the war, because it revealed the apparent weakness of the Japanese - apart from the breaking of the Japanese code, the Americans could recover and reuse more ships than the Japanese, while most Japanese ships were wrecked.\\
\\
In 1944, the Americans fight to retake the Philippines. There's an iconic photo of MacArthur walking up the beach against the backdrop of amphibious assault boats, with a look of grim determination on his face - it blew up because something something American glory and what not. What \textit{actually} happened was that MacArthur did not want to get wet, but because the boat he was on could not land, he had to walk. This grim determination was just him being pissed as hell.\\
\\
The Battle of Luzon - Japan reinstated Yamashita to defend Manila. Yamashita had the same idea as the Americans - a sort of open city, and planned to regroup. However, there was insubordination - Rear Admiral Iwabuchi had a vendetta against the Americans, and held Manila fiercely, while massacring the civilians (Manila Massacre - 100,000 to 250,000 dead). The Americans bombard Manila to hell, killing thousands upon thousands of civilians. Even libraries were destoyed - essentially, Filipino history ends in 1945 because everything was destroyed.\\
\\
There's also some serious \textit{Force 136}/ MPAJA styled shenanigans - the \textbf{Hukbalahap}. A communist/socialist guerilla movement that fought the Japanese \textit{and also the Americans}. At their peak, they had 50,000 to their name. They continue fighting the Philippines Government until 1946 before being put down by the president. In 1949, they go a bit crazy and assassinate Quezon's wife, which kind of pisses off the Americans.
\end{document}