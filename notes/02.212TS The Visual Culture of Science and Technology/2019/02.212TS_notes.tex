\documentclass[a4paper]{article}

\usepackage[margin=1in]{geometry} 
\usepackage[hidelinks]{hyperref}
\usepackage{bookmark}
\usepackage{tocloft}
\renewcommand{\cftsecleader}{\cftdotfill{\cftdotsep}}
\renewcommand{\baselinestretch}{1.25} 

\usepackage[T1]{fontenc}
\usepackage{kpfonts}
\usepackage{csquotes}


\usepackage{enumitem}

\usepackage{graphicx}
\graphicspath{ {./images/} }

\setcounter{tocdepth}{4}
\setcounter{secnumdepth}{4}

\usepackage{amssymb}
% \usepackage{amsmath}

\usepackage{cellspace}
\setlength\cellspacetoplimit{4pt}
\setlength\cellspacebottomlimit{4pt}
\usepackage{framed}

\usepackage{xcolor}

\usepackage{float}

\title{Notes on Course Readings and Class Recitals\\[0.1cm]
	\large 02.212TS The Visual Culture of Science and Technology, Fall 2019}
\author{Tan Tiang Teck \and Tey Siew Wen}
\date{06 Dec 2019}

\begin{document}
	\maketitle
	\tableofcontents
	\newpage
	
\section{W1: Visualizations in Science and Social Science}
\subsection*{Weekly Readings}
\begin{itemize}
	\item Michael Lynch, ``Chapter 2: The Production of Scientific Images: Vision and Revision in the History, Philosophy, and Sociology of Science,'' in Luc Pauwels, ed., \textit{Visual Cultures of Science: Rethinking Representational Practices in Knowledge Building \& Science Communication} (Hanover, NH: University Press of New England,
	2005), pp. 26-40.
\end{itemize}
\subsection{Notes on Readings}
\subsubsection{The Meaning Of Visuals}
\begin{itemize}
	\item Pictures themselves do not mean anything. It is the practices through which the scientific phenomena is visualized that give meaning to the depiction.
	\begin{itemize}[label=$\circ$]
		\item \textbf{Shapin and Schaffer (1985)}: Virtual Witnessing - readers are given a sense of having experienced the events described in a report with visualizations
	\end{itemize}
	\item Visualization is not just about the sense of sight, it is a language. \\We need to consider factors such as:
	\begin{itemize}[label=$\circ$]
		\item The source of data
		\begin{itemize}[label=\tiny$\blacksquare$]
			\item The type of instrument used to capture the data
			\item Who managed the data
		\end{itemize}
		\item Format of data
		\item Platform where the data is showcased
		\item \textbf{e.g. Martin Rudwick (1976)}: Visual Language developed by the 19th century geology to increase legibility of the Earth.
	\end{itemize}
	\item Displays are not merely perceptoral; they are complex assemblages of verbal, numerical, geometrical, textual, material, instrumental, and pictorial phenomena.
	\item Visualizations are not always photos; they are often not exact replicates of what its representation look like in reality
\end{itemize}

\noindent Other less addressed points by social scientific researchers:
\begin{itemize}
	\item Facts are all artefacts.
	\begin{itemize}[label=$\circ$]
		\item The term "artefact" often is used in reference to an inadvertent product of laboratory procedures that is separate from phenomena of interest.
		\item But at the same time, not all artefacts are unnatural in origin
	\end{itemize}
\end{itemize}
\newpage
\subsubsection{Influential philosophies of science on current sociology and history of science}
\begin{itemize}
	\item \textbf{Hanson (1967), Kuhn (1970)}: analyze observational activities in terms of individual perception and cognition
	\begin{itemize}[label=$\circ$]
		\item Hanson said that \textbf{observation is theory laden}, but this is too crude a formulation to describe the various ways in which objects become identity laden, category bound, and graphically embedded.
	\end{itemize}
	\item \textbf{Polanyi (1967)}: treats observational instruments as extensions of individual sense organs through which the skilled scientist learns to see or feel what the instrument discloses or contacts.
	\item \textbf{Feyerabend (1975)}: interprets visual representations as evidence of historically and culturally relative cognitive and perceptual tendencies
\end{itemize}
\subsubsection{Other interesting thoughts}
\begin{itemize}
	\item \href{http://jalammar.github.io/about/}{\textcolor{blue}{Visualizing ML}}
\end{itemize}
\subsection{Notes on Class Recitals}
\subsubsection{Introduction}
\begin{itemize}
	\item More studies into visual images as a result of interest in the contents of scientific texts, discourse and practices
	\item The ideas and principles behind the images more important than the images themselves: the pictures are a means to an end
	\item What should studies into visualisation in science contain? Why useful for understanding scientific practices?
	\begin{itemize}[label=$\circ$]
		\item Pictorial documents used as evidence of research and used in publications to demonstrate said research
		\item Visualisation is a part of the scientific method and a part of getting others to understand your research
		\item Inseparable from scientific research
	\end{itemize}
\end{itemize}
\subsubsection{The concept of visualisation in science}
\begin{itemize}
	\item While it is easy to say that visualisation in science boils down to making observations and measurements into standardised and universally understood graphs and charts, there is no one unified, singular method of representation
	\item The method(s) of representation chosen depends on context and practicality
	\item Visualisations are just representations, and are not to be taken as sense-perception
	\item Hanson and Kuhn look at measurements in terms of the scientists’ perceptions
	\item Polanyi says that measurement devices are extensions of the scientists’ senses
	\item Feyerabend says that visual representations are a result of how people tend to think of and perceive scientific phenomena, taking into account their history and culture
	\item Visualisation is not perception: while perception can be augmented by technology, you cannot dismiss visualisation as just the perception of something
	\item Visualisation allows for individual perception and judgement
	\item Instead of scientific discoveries being a largely personal and instantaneous affair, there is a process of repeated refining between scientists to produce a set of findings that were presentable to the rest of the world
	\item Sometimes it is good to think of visualisation as a way to elucidate phenomena that is hidden from view
	\item Some figures induce understanding of concepts in readers
	\item Others visualise information encoded in text: translations rather than to illustrate as in the previous point
\end{itemize}
\subsubsection{How to make displays}
\begin{itemize}
	\item Sometimes it be like arts and crafts
	\item Sometimes it be very sanitised and standard, like factory workers who don’t actually know what they’re doing, but they know how to do the one simple but important task along the assembly line
	\item Maybe cannot say that making visualisations is like making arts and crafts, because sometimes the visualisations are not accurate
	\item Latour and Woolgar say that whatever scientists know about their objects is a consequence of how they represent them: they perceive of an object through the way it is presented to them by their methods of observation
	\item Distinctions between representative and nonrepresentative artefacts collapse in postmodern conceptions of science
	\begin{itemize}[label=$\circ$]
		\item When you assert something is scientifically true, there is always uncertainty. What if your data is fuck? What about flaws inherent in the methodology of the data collection? What if those aren’t actually flaws or outliers to be ignored but actually some new shit you didn’t predict/know existed?
		\item When preparing samples for observation, what if the prep work results in a different result than if the work was not done?
		\item Furthermore, the way you represent your data is ostensibly different from what the samples you used to gather this data are
	\end{itemize}
\end{itemize}
\subsubsection{Enframing}
\begin{itemize}
	\item Heidegger
	\item An all-encompassing view of technology as how humans exist as being humans, where through technology, things IRL are no longer regarded as objects, but are things dissected and boiled down to how they benefit humanity. While technology has revealed some part of the object, it has lost the forest for the trees
	\item Lab and field observations are different: one is more mathematical, the other is more descriptive
	\item Pragmatic and semiotic considerations are more interesting
\end{itemize}

\section{W2: Making Things Visible}
\subsection*{Weekly Readings}
\begin{itemize}
	\item Bruno Latour, ``Visualization and Cognition: Drawing Things Together,'' in H. Kuklick, ed.,\\ 
	\textit{Knowledge and Society, Studies in the Sociology of Culture Past and Present}, Jai Press vol. 6, pp. 1-40.
\end{itemize}

\subsection{Notes on Readings}
\subsubsection{Prof Samson's introduction}
\begin{enumerate}
	\item It introduces a number of important concepts from science and technology studies that can be used to think about visual culture and how knowledge is formed and how it operates.
	\item It touches on the production of images (how they are made matters, for Latour). This is the first dimension (visualization) I wanted you to think about images through, as mentioned in week 1.
	\item It discusses images in terms of their relation to other images, or as part of a network of images (and thus things and people). This is the second dimension (network) I wanted you to think about images through, as mentioned in week 1.
\end{enumerate}
\subsubsection{Thinking with eyes and hands}
\begin{itemize}
	\item Modern scientific culture is not born from some random magic stuff that happens in people’s brains
	\item It is not a spontaneous and instantaneous process, because Occam’s razor
	\item It is a very gradual one
	\item The dichotomy between supposed modern science and classical science is tenuous, but there are good reasons for their existence
	\item Because the effects of the two differ so much, you cannot possibly compare classical science to modern science without sounding ridiculous. It is easier to explain such big differences with an equally big reason
	\item Accepting the dichotomy dodges the baggage that comes with taking a relativist approach
	\item According to Latour, Visuals explain almost everything or almost nothing about the idea that it was meant to convey. 
\end{itemize}
\noindent\textbf{Definitions:}
\begin{itemize}
	\item \textit{Syllogism:} A form of reasoning in which a conclusion is drawn from two given or assumed propositions (premises), where there is a common term/concept is present in the two premises but not in the conclusion. (so the conclusion may be invalid)
\end{itemize}
\noindent\textbf{Interesting quotes}
\begin{itemize}
	\item ``If we wish to speak of a `savage mind', these are some of the instruments of its domestication'' (Goody, 1977)
	\item ``How much explanatory burden can a visual truly carry? How many cognitive abilities may be, not only facilitated, but thoroughly explained by them?'' - Latour
\end{itemize}
\subsubsection{Big problems with mundane solutions}
\begin{itemize}
	\item Instead of thinking of grandiose reasons for the sudden success of modern science, we should find something more mundane
	\item But how mundane is mundane?
	\item If not mental causes, why not material? Maybe ECONOMICS can explain why: more resources make it easier for scientific discoveries and developments?
	\item But economics is also science and if we take economics to be the origin of modern science, then the economic theories that contribute to this scientific success are not falsifiable and so cannot be taken as absolute truth
	\item BUT WHAT IS TRUTH EVEN
	\item Latour studies biologists. He says that we should look at how information and findings are documented, the practice of recording stuff down
	\item But while the popularisation of print and visualisation is a necessary condition for modern science, it is not sufficient because it seems too trivial
	\item Looking further, we look at what situations where we can expect improvements in writing and imaging can change the way we think
	\item In differences of opinion between learned minds, the side that wins is most often that garners the most allies the most quickly. Documentation helps that process because people are more easily swayed if they are ABLE to understand your POV and happen to agree with it. Documentation facilitates UNDERSTANDING
	\item It is not the attributes of the writing or imaging, but the way they facilitate mutual understanding that makes dedicated allies. When more people believe in an opinion, the stronger the basis for it being generally accepted as factual truth, and the more supporters there are to support claims of originality from the first author
	\item La Perouse’s journey shows that advancements in documentation technology and the fact that documentation facilitates transmission of information are key drivers of modern scientific progress
	\item It also tells us more about the requirements to facilitate this transmission of information:
	\begin{itemize}[label=$\circ$]
		\item Mobile: bring the info to others
		\item Immutable: cannot deteriorate while you are transporting it
		\item Presentable and readable: how will you transmit information if other cannot access it mentally?
		\item Combinable: Make new stuff out of old stuff so it spreads
	\end{itemize}
	\item Latour says that the human mind is in the process of being domesticated: prior to a few key inventions, intellectual discourse is disordered and fragmented. Something needs to train the human mind to focus on what is important instead of wandering
\end{itemize}
\subsubsection{Immutable Mobiles}
\begin{itemize}
	\item Optical consistency
	\begin{itemize}[label=$\circ$]
		\item The invention of perspective in visual recordings
		\item Even though visually, physical quantities can change, perspective allows viewers to recognise that these quantities do not in fact, change but are merely a consequence of the way the medium was created
		\item The creation of such a standard in longitude, latitude and geometry allows continuity of visual representation across space: a building in Rome can be understood and replicated perfectly, material factors allowing, in somewhere across the world
		\item Sight happens to be especially important because it is difficult to replicate aural, olfactory and tactile fidelity in comparison; this has made sight a foremost sense compared to the rest
		\item Link back to mobile immutability
		\item Perspective also allows fictional objects to be depicted realistically, normal objects to be depicted fantastically - mixing! With realistic enough depictions, even fantasy can look real
	\end{itemize}
	\item Visual Culture
	\begin{itemize}[label=$\circ$]
		\item ``How a culture sees the world and makes it visible''
		\item The restrictions of time and space are greatly reduced
		\item Perspective plus printing press plus naval mobility is important combination because books can now reliably and easily have pictures alongside words: people no longer have to rely solely on their imagination
		\item Changes in graphism: “the expression of thought in material symbols”
		\begin{itemize}[label=\textsection]
			\item There is a lag time between the introduction of the printing press and the beginning of `exact realistic pictures'
			\item This is because people print stuff first. They then realise that there are discrepancies and that there is a need for accuracy
			\item Prior to print, distinct disciplines had their own developments and intellectual feats that stayed localised and temporary just because the lack of mobility and immutability meant that any advancements could easily be undone. But with print allowing them to transcend the limits of space-time, the disciplines could mingle and advance without accumulation of error
			\item Maxwell’s demon is a thought experiment in which the 2nd law of thermodynamics could possibly be violated by reducing entropy through use of impossibly quick reaction speeds
		\end{itemize}
	\end{itemize}
	\item A new way of accumulating time and space
	\begin{itemize}[label=$\circ$]
		\item The printing press is some powerful stuff: it improves both mobility and immutability where previous advancements only improved either one
		\item Replicability and repeatability
	\end{itemize}
\end{itemize}
\noindent An example of immutable mobile: MONEY 

\subsubsection{Capitalizing Inscriptions to Mobilize Allies}
\noindent\textbf{Characteristics of Inscriptions}
\begin{enumerate}
	\item \textit{Mobile:} The subjects of interest are often not (easily) transportable to be shown physically to people outside its natural environment
	\item \textit{Immutable:} The subjects of interest are still in the state that they were desired to be in by the presenter
	\item \textit{Flat:} No hidden dimensions
	\begin{enumerate}[label=\alph*.]
		\item Indexing and inventorising, searching and updating
		\item Correct mistakes in your docs
		\item Good storage
		\item Good summary, easier to digest and understand
		\item Consistency because the curator presents everything he feels is important in the documentation
		\item Flattening objects makes it easier to internet to ppl
	\end{enumerate}
	\item \textit{Scale:} May be modified at will to a size that can be easily viewed fully by the target audience.
	\item \textit{Easily Reproducible:} Low cost of replication, allowing widespread representation of the subject
	\item \textit{Combination \& Superimposition:} Diff scaled images can be combined together to be viewed simultaneously; a feat that cannot be done otherwise without visualization.
	\item \textit{Optical Consistency:} Everything can be converted into diagrams and numbers, and combination of numbers and tables can be used which are still easier to handle than words or silhouettes (Dagognet, 1973). 
	\begin{enumerate}[label=\alph*.]
		\item ``You cannot measure the sun, but you can measure a photograph of the sun with a ruler.''\\ - Latour
	\end{enumerate}	
\end{enumerate}
\noindent Knowing the benefits of visualization, any new invention that serves as an instrument to enhance these characteristics immediately becomes popular among scientists to better express the findings of their research.
\subsubsection{Paperwork}
\begin{itemize}
	\item Realms of Reality: mechanics, economics, marketing, scientific organization of work
	\begin{itemize}[label=$\circ$]
		\item These can be all flattened out onto a single surface in industrial drawings, allowing work to be planned, dispatched, realized and attribute responsibility accordingly.
		\item Gives rise to Metrology: The study of measurement
	\end{itemize}
	\item All innovations in picture making, equations, communications, archives, documentation, instrumentation, argumentation, will be selected for or against depending on how they simultaneously affect either inscription or mobilization.
	\begin{itemize}[label=$\circ$]
		\item \textit{E.g. of Taking Notes: }Usually in doc, plain text, but now people prefer markdowns more as it is more convenient to add formatting like headers, bold, bulleted lists, and so on. 
	\end{itemize}
	\item The power of a person/organisation is not absolute, but rather relative with the ability to produce, capture and sum up information about other places and time. (Callon and Latour, 1981)
\end{itemize}
\noindent\textbf{Interesting quotes}
\begin{itemize}
	\item ``A man is never much more powerful than any other --- even from a throne; but a man whose eye dominates records through which some sort of connections are established with millions of others may be said to \textit{dominate}'' - Latour
\end{itemize}
\subsection{Notes on Class Recitals}
\subsubsection{Prompts for in-class activity}
\begin{itemize}
	\item What is one example of an immutable mobile?
	\item Who makes it?
	\item Who uses it?
	\item What information is it supposed to represent?
	\item What network is it a part of?\\(That is, what other things, images, texts, or people is it connected to?)
\end{itemize}
\subsubsection{Teck's thoughts}
Centre of calculation\\
But Chinese also got documents?\\
Angmohs have larger network to discuss and stuff but not for China\\
This is because of PRINTING\\
One critique: interpretation. No method of information transmission is perfect, they are all subject to interpretation\\
Another critique: why did the angmohs travel but chinese stop travelling?\\
If picture and flatten too stronk, then what if the only thing we see is these pics and not the true reality? - Socrates
\subsubsection{Siew Wen's thoughts}
Meterology - the study of measurement
\begin{itemize}
	\item Goethe: Goethe Barometer
\end{itemize}
\subsubsection{Prof Samson's thoughts}
Key concepts that you should make yourself familiar with from his essay include, but are not limited to:
\begin{enumerate}
	\item \textbf{Immutable mobile. }The simplest definition of this term I can think of is from page 7 of his essay: ``… objects which have the properties of being mobile but also immutable, presentable, readable and combinable with one another.'' For example, a map on paper can be an immutable mobile because it can move and the information it is supposed to present or capture does not change. That is, any object that does what he calls ``gathers and presents'' information to other people across space and time.
	\item \textbf{Inscription.} This is basically another term for an immutable mobile, though it also refers more broadly to the process of making an image or object that is meant to convey information. The term to inscribe, for example, brings to mind the physical process of transferring information from the world to paper or from one medium to another. He gives us nine important characteristics of inscriptions beginning on page 20. You can look at them and think about why each characteristic is important for creating networks, and thus knowledge.
	\item \textbf{The `dichotomies.'} The basic question that Latour is trying to address is ``What is the difference between modern/not-modern, east/west, and science/pre-science?'' He is interested in analyzing the validity of the categories (modern, traditional, etc.) he sees as structuring the way many people think about the world. What is his conclusion, or answer, to this question?
	\item \textbf{Centres of Calculation:} The notion of `centre of calculation' was developed by the French sociologist Bruno Latour (1987). It is a concept about the venues in which knowledge production builds upon the accumulation of resources through circulatory movements to other places (from \href{https://repository.lboro.ac.uk/articles/Centre_of_calculation/9486431}{\textcolor{blue}{https://repository.lboro.ac.uk/articles/Centre\_of\_calculation/9486431}}). In other words, it is a place where inscriptions are stored and where they can be accessed and used for various reasons. 
\end{enumerate}
\noindent Fact: See his discussion on page 5. Information becomes `fact' if it is aligned within a network of accepted practices, standards and so on. He calls this having faithful allies. Think about what this might mean in terms of what we know about the world.

\newpage
\section{W3: Network of Images}
\subsection*{Weekly Readings}
\begin{itemize}
	\item Michael Lynch, ``Discipline and the Material Form of Images: An Analysis of Scientific
	Visibility,'' \textit{Social Studies of Science}, Vol. 15., No. 1 (1985), pp. 37-66.
\end{itemize}

\subsection{Notes on Readings}
\subsubsection{Prof Samson's introduction}
\begin{itemize}
	\item Science have many pictures very cool, can see what you normally cannot
	\item When you can see you can use math and analyse
	\item But to visualise must use technology and this makes it fake cuz you rely on tech and not your own senses
	\item In science, man use conventions to make visualisations: these conventions reflect the disciplinary organisation of scientific labour and reflect the organisation of nature
	\item Lynch wants to characterise the features of the production of scientific displays
	\item Everything that happens in the lab is part of a very interconnected network that outputs displays
	\item Lynch wants to present an account of how the specimen is transformed into visible and analysable data through the various stages of a project
\end{itemize}

\subsubsection{Proto-scientific example}
\begin{itemize}
	\item The figure shown is some simple map displaying the home ranges of marked lizards.
	\begin{itemize}[label=$\circ$]
		\item Very easy to understand, even by laymen
		\item This makes it easier to identify common elements of more complicated representations
		\item The figure is part of a set of instructions for preparation for systematic observation and experimentation, not instructions for actual experimentation
		\item Proto-scientific because no hypotheses are tested
		\item The instructions are in a way a method of turning the landscape into a geometricized workplace
		\item The instructions say to amputate lizard toes to mark them. In doing so, you are reducing the lizard to just a numbered sample
		\item Numbered stakes placed at regular intervals on the location outlined by the map reduce the location to something mathematical
		\item Selecting the spatial interval between the stakes in a manner that corresponds to lizard territory sizes presupposes some aspect of lizard behaviour
		\item The surface of the map combines the grid and the actual landscape. It allows for the manipulation and utilisation of an ideal
		\item The stakes are a sort of `proto-map' from which the actual map on paper originates
		\item Each specimen is reduced to a dot on the map and sightings over a period of time are represented spatially on the map: anything over than these 2 features are effectively omitted
	\end{itemize}
	\item How does science determine what is natural based on the graphics created in the process of science?
	\item The above lizard illustration shows a general workflow: marking subjects, preparing the medium to contain the information to be expressed and the normalisation of subjects to omit features not used in that particular scientific enquiry. The test subjects become “docile objects”.
\end{itemize}

\subsubsection{Axon sprouting}
\begin{itemize}
	\item Axons are the tentacles coming out from nerve cells that connect to other cells, through which they send impulses
	\item Marking, constituting graphic space and normalizing observations
	\item Scientists make and analyse pictures of rat hippocampus, put together to see the structure of the brain
	\item Experiment to see how adjacent axons layers would react to physical trauma to an axon layer
	\item Each picture was one sector of a grid and counts of intact axon terminals were counted and plotted on graph
	\item Results show that axon sprouting, a mechanism to partially compensate for irreversible tissue damage is a thing
	\item Now that there is a link between the graph and the collected test samples, we can see how the material was successfully 'geometricized', 'chronologized' and 'mathemati- zed' such that its scientific visibility and demonstrability was achieved
	\item Which brings us back to the themes of marking, constituting graphic space and normalising observations
\end{itemize}

\subsubsection{Marking}
\begin{itemize}
	\item Color coded marks were hand drawn on the montages 
	\item Indexing: Ink slashes were drawn on the specimens’ body 
	\item Labelling: Stains applied to tissues to make them visible 
	\begin{itemize}[label=$\circ$]
		\item Cellular structure was dependent upon the devices of ‘labelling’ for its accountability
		\item The stain defined, pointed to and consolidated the visibility of its object.
		\item However, staining might obscure the object itself or create fictional resemblances; constituting ambiguities in the form of the object vs the form of the label
	\end{itemize}
\end{itemize}

\subsubsection{Upgrading Visibility}
\begin{itemize}
	\item Achieving Visibility is a process of Selective Perception
	\item The number of marked terminals on the surface of the montage did not correspond to the actual number of terminals, nor could they.
	\begin{itemize}
		\item The visible terminals on the montage were marked only because they were counted as “good enough” to be visualized
		\item The rest were ignored or are indistinct.
	\end{itemize}
	\item Use of electrophysical monitors for detecting electrochemical events, which are faint and otherwise undetectable.
	\item Unlike a photograph, a drawing can superimpose, in one continuous line, courses of dendrites and axons which snake in and out of the focal field of the microscope: A series of adjustments of focus can be collapsed into one schematic trace
\end{itemize}

\subsubsection{Constituting Graphic Space}
\begin{itemize}
	\item Practices for constituting graphic space are pre-linguistic modes of order production.
	\item In the axon sprouting example above, Geometries were appropriated both from: 
	\begin{enumerate}
		\item 'endogenous' conformations of the cellular material examined 
		\item 'exogenous' conformations of the instrumental fields and
		\item Literary formats used in rendering the materials examinable
	\end{enumerate}
	\item The sequence of steps carried out during the project seem like a bootstrap operation to upgrade the visible form of ‘endogenous’ geometries that are aligned with ‘exogenous’ graphic formats.
\end{itemize}

\subsubsection{Exposed Geometries}
\begin{figure}[H]
	\centering
	\includegraphics[width=0.5\linewidth]{images/W3Fig3}
	\label{fig:w3fig3}
\end{figure}
\begin{itemize}
	\item Figure 3 provides scale of distance along the vertical axis of the graph by tracing the origin to a line which intersects the cell body of the granule cell.
\end{itemize}
\begin{figure}[H]
	\centering
	\includegraphics[width=0.7\linewidth]{images/W3Fig4}
	\label{fig:w3fig4}
\end{figure}
\begin{itemize}
	\item Figure 4 traces the linear conformation of the granule cell layer from a 'slab' of tissues extracted cross-sectionally from a section of the hippocampus.
	\begin{itemize}[label=$\circ$]
		\item The tissue slices exposed the cell layer and excised it in such a way as to place the lower frame of the rectangular slice in a parallel alignment with a relatively straight segment of the curving granule cell layer (second from left drawing in Figure 4). 
		\item The slab was cut so that its edges located a parallel alignment of pre-delineated anatomical structures.
		\item UNIFORMITY is the basis for visualising Geometry
	\end{itemize}
\end{itemize}

\subsubsection{Upgrading Geometricity}
\begin{itemize}
	\item The anatomical layout of the slab
	\begin{itemize}[label=$\circ$]
		\item Y: the dendrites of the granule cells
		\item X :the axons synapsing with those dendrites as horizontal components
		\item Each linear slice instrumentally constituted planes and lines of orientation referenced, with a superior 'exactness', to the exposed-linearities of the anatomical divisions
	\end{itemize}
	\item \textit{“The thin sections were constructed as if they were pages of brain tissue to be read for their graphic imprint”}
	\begin{itemize}[label=$\circ$]
		\item Tissue choppers and microtomes were used, respectively, to slice thick and thin sections in preparation for microscopic viewing.
	\end{itemize}
	\item The uniformity of the tissue grain was accentuated not only through the imposition of instrumental planes of operation and manually inscribed lines, but also the selection of sites for shooting the series of micrographs expropriated locales
\end{itemize}

\subsubsection{Utilizing Formats to Compose Scales}
\begin{itemize}
	\item Lines and planes provide linear edges for the location of scales 
	\item Flat surfaces for the superimposition of grids
	\item Discrete sites are points within the grid.
	\item Proto-mathematical properties of objects are selectively exposed and are carefully placed to construct the material form of visible displays
\end{itemize}

\subsubsection{Normalizing Observations}
\begin{itemize}
	\item Each marked terminal is the product of a practical course of actions which prepared the specific terrain for visible assessment and measurement before including the specific case into the corpus of acceptable observations.
	\item The graphic display normalizes the properties of each animal and each counted terminal
	\item Graphics preserves the aggregate properties of the marked cases and drops any direct reference to the singularity
	\newpage
	\item The object of interest becomes both more than, and less than itself.
	\begin{itemize}[label=$\circ$]
		\item E.g. The specimen ‘animal’ is more than just a creature living out its life in the cage, since in the Figure you can now see all its insides. It became less than the animal since the original animal exterior is kind of thrown away in favour of the residues retained for inspection. 
		\item \textit{“The animal is ‘sacrificed’ along with a cohort of others for the sake of an aggregate line on the graph”}. $\to$ All its struggles that it went through etc as a living being to obtain the scientific result is ignored.
	\end{itemize}
	\item The chart is an idealized account of the lab’s work.
\end{itemize}

\subsubsection{Methodology as an Externalized Retina}
\begin{itemize}
	\item Graphical formats, instrumental fields \& preparatory techniques in histology operate like elements of an externalized retina: activate the perceptible \& schematically processing it. 
	\item Retinal Cells constitute our visual world by detecting lines, edges and coded ranges of color.
	\item The instrument does not merely extend the sensitivities of sensory perception; it reconstructs the world that is seen.
	\item Lab data is neither wholly constructed nor simply a \textit{mirror of nature} arising from an encounter between a rational mind and the inherently orderly nature. This is Social Studies of Science.
\end{itemize}

\subsection{Notes on Class Recitals}
\subsubsection{Siew Wen's thoughts}
\begin{itemize}
	\item People validated the results of technology through their senses before relegating their own responsibilities to machines 
	\item In marking, the medium used for differentiating individuals was the specimen’s body, a “mutable mobile”.  
\end{itemize}

\subsubsection{Teck's thoughts}
\begin{itemize}
	\item In creating docile objects, some smidge of subjectivity might sneak in: the stuff you choose to look at, the normalisation process, etc.
	\item Images created in the scientific process are not necessarily representative of the object they are supposed to depict
	\item Scientific facts are a hybrid object, a combination of an actual object and the processes that make a certain aspect of it visible, the process that leads up to the fact
	\begin{itemize}[label=$\circ$]
		\item It’s not just the thing itself, but the thing through the lens of whoever came up with the fact / prepared the materials
	\end{itemize}
	\item Anonymous badger
\end{itemize}

\subsubsection{Prof Samson's thoughts}
\noindent\textbf{Rendering}
\begin{itemize}
	\item In order for scientists to study and analyze natural objects, like lizards or rat brains, they first must be made visible in a material form. This can be a photograph, a diagram, a chart or table. The point is, that scientists must have something the can see and work with, or manipulate, in order to generate scientific knowledge.
	\begin{itemize}[label=$\circ$]
		\item For example, the lizards in his article are transformed into a numerical formula: RF4-RH3-LH5. This code can then be put in a table or chart or other visualization. That is, the real lizard in the world cannot be studied directly, but the code can be.  
	\end{itemize} 
	\item The process of making a natural object visible and material is ‘rendering.’ In other words, you can think of this term as simply referring to the various operations that are necessary to make a natural object into a visual image or other object. 
	\item During the rendering process, many different things can happen. He names a few, like marking, constituting a geographic space, upgrading geometricities, and so on. What is important to remember is that different transformations take place during a rendering process. Sometimes they involve marking, sometimes not. It depends on the image being created. 
\end{itemize}
\noindent\textbf{Docile Object}
\begin{itemize}
	\item The output of the rendering process is what he calls a docile object. This can be something like a photograph or chart. This object is ‘docile’ because it can be manipulated and analyzed in ways that the ‘wild’ object (the lizard in the forest) cannot be. Hence, he says that scientists work on docile objects.  
\end{itemize}
\noindent\textbf{Fact/Artifact}
\begin{itemize}
	\item Scientific facts are thus the product of analyses on what I call ‘hybrid objects.’ Facts about the world are neither “wholly out there” in the world nor “wholly constructed out of thin air.” What we know about the world depends on the practices, technologies, and assumptions we have about the world.
\end{itemize}

\newpage
\section{W5: Technology and Objectivity}
\subsection*{Weekly Readings}
\begin{itemize}
	\item Peter Galison and Lorraine Daston, ``Image of Objectivity,'' \textit{Representations}, No. 40
	Special Issue: Seeing Science (Autumn, 1992), pp. 81-117.
\end{itemize}

\subsection{Notes on Readings}
\subsubsection{Summary}
This essay is an account of the moralization of objectivity in the late nine­teenth and early twentieth centuries as reflected in scientific image making. We will use scientific atlases from diverse fields (anatomy, physiology, botany, paleontology, astronomy, X-rays, cloud-chamber physics) and from a span of several centuries (eighteenth to twentieth) to chart the emergence and nature of new conceptions of objectivity and subjectivity.

\subsubsection{The Talismanic Image}
\paragraph{Concepts by French Physiologist E.J Marey}
\begin{itemize}
	\item Before science, language is often inappropriate to express exact measures or definite relations.
	\item Graphical Expression is the language of phenomena themselves → Let nature speak for itself
\end{itemize}

\paragraph{Modern Objectivity}
\begin{itemize}
	\item M. objectivity mixes rather than integrate disparate components of objectivity
	\begin{itemize}[label=$\circ$]
		\item E.g. of objectivity: non-interventionist or mechanical etc.
		\item Historically and conceptually distinct
		\item Each of the several components of objectivity opposes a distinct form of subjectivity
	\end{itemize}
	\item M. objectivity can be applied to everything from empirical reliability to procedural correctness to emotional detachment
	\item M. objectivity is somewhat an embodiment of the negative character of all forms of objectivity
\end{itemize}

\paragraph{Mechanical Objectivity}
\begin{itemize}
	\item Negative: Aims to eliminate the mediating presence of the observer
	\begin{itemize}[label=$\circ$]
		\item Disparage the senses that register the phenomena
		\item Ward off theories \& hypotheses that distort the phenomena
	\end{itemize}
	\item Requires observations to be endlessly repeated
	\item The essence of self-discipline - removing any possible bias from the individual and concentrating only on results that are produced
	\begin{itemize}[label=$\circ$]
		\item Positive: Glorify reliability of the bourgeois rather than the moody brilliance of the genius
		\item Moralized Vision: Self-command triumphing over the \textbf{temptations and frailties of flesh \& spirit}. These flaws can be simplified as
		\begin{itemize}[label=\tiny$\blacksquare$]
			\item the inevitable lack of true sight - We might see \textit{x} as \textit{x\_with\_bias} instead of just that \textit{x}.
			\item Intentional/ unintentional tampering with \textit{facts} in writing 
		\end{itemize}
	\end{itemize}
\end{itemize}

\paragraph{Mechanized Science vs Moralized Science}
\begin{itemize}
	\item In the 19th century, commonplace for Machines to be paragons of certain human virtues
	\item \textit{As the photograph promised to replace the meddling, weary artist, so the self-recording instrument promised to replace the meddling, weary observer.}
	\begin{itemize}[label=$\circ$]
		\item The instruments not only save human labor but also surpass human observers in laboring virtues
	\end{itemize}
	\item Machines have Freedom from will vs Humans Freedom of Will
	\begin{itemize}[label=$\circ$]
		\item Without theory and judgement, there is no way for machines to be intervening in Science
		\item Negative Ideal of non-interventionist objectivity with its morality of restraint and prohibition
	\end{itemize}
\end{itemize}

\subsubsection{Truth to Nature}
\paragraph{Typus/Archetype}
\begin{itemize}
	\item Representation of an object in general morphological structure, and not any particular object.
	\begin{itemize}[label=$\circ$]
		\item Other Objects can inherit from this typus
	\end{itemize}
\end{itemize}
\paragraph{Atlas}
\begin{itemize}
	\item Practitioners of the sciences prepared editions of their designated phenomena in the form of \textbf{atlases}
	\begin{itemize}[label=$\circ$]
		\item The purpose of atlases was to standardize observing subjects and observed objects of the discipline by eliminating idiosyncrasies - of both the individual observers and also those of individual phenomena
	\end{itemize}
	\item Nature seldom repeats itself, variability and individuality in nature is the rule rather than the exception.
	\begin{itemize}[label=$\circ$]
		\item \textit{Nature is full of diversity, but Science cannot be.}
	\end{itemize}
	\item Make nature safe for science: Replace raw experience with digested experience
	\item Latour’s characteristic of immutable mobile
	\item In domains of science such as pathology as compared to anatomy, the opportunities for working specimens are considerably rare and fleeting.
	\item Concern for objectivity $\neq$ Concern for accuracy
	\begin{itemize}[label=$\circ$]
		\item These concerns may even conflict when it comes to mechanical objectivity, as it undermines the primary goals of atlases in representing nature
	\end{itemize}
	\item Conflict between \textit{truth to type \& truth to individual specimen}
	\begin{itemize}[label=$\circ$]
		\item ○	Dangers of excessive accuracy: Outliers sometimes are reported by illustrators, which may undermine the efforts of scientists that have mined their data
	\end{itemize}
\end{itemize}

\paragraph{Interesting Facts}
\begin{itemize}[label=$\circ$]
	\item Why a book of maps is called an atlas: The term “atlas” comes from the name of a mythological Greek figure, Atlas. As punishment for fighting with the Titans against the gods, Atlas was forced to hold up the planet Earth and the heavens on his shoulders. Because Atlas was often pictured on ancient books of maps, these became known as atlases.
\end{itemize}

\paragraph{Ideal vs Characterised}
\begin{itemize}
	\item Ideal: Perfect, Characterised: Typical
	\item \textit{``Whenever the artist alone, without the guidance and instruction of the anatomist, undertakes a drawing, a purely individual and partly arbitrary representation will be the result, even in advanced periods of anatomy … however under the supervision of an expert anatomist, it becomes effective through its individual truth, its harmony with nature, not only for purposes of instruction but also for the development of anatomic science.'' - Ludwig Choulant, 19th century historian of anatomical illustrations}
	\begin{itemize}[label=$\circ$]
		\item Beauty of depiction can be part and parcel of achieving accuracy, not a seduction to betray it
	\end{itemize}
\end{itemize}

\paragraph{Objectivity and mechanical representation}
\begin{itemize}
	\item Photography make representation good, but debate about whether it was completely objective never disappeared
	\item People still think you need mechanical or procedural safeguards against subjectivity
	\item Willian Andersen's ``introductory address'' to the Medical and Physical Society of St. Thomas Hospital
	\begin{itemize}[label=$\circ$]
		\item Medicine no longer relying on artist depictions
		\item \textbf{Artists might be a liability} in fact: they are too artsy and add fancy stuff cuz art
		\item Instead of relying solely on artistry, medicine needs to move towards using art to create accurate depictions. \textbf{Science-directed art}
		\item Less manual work needed $\Rightarrow$ less room for interpretation
	\end{itemize}
	\item Strict policing against subjectivity: partial application of photography. Through scrutiny and making sure the artists followed only and only what was on the photos they were shown instead of the actual subject
	\begin{itemize}[label=$\circ$]
		\item Sabotta and his anatomical drawings:
		\begin{itemize}[label=\tiny$\blacksquare$]
			\item Photographs gathered and used as basis for drawings, but emphasised that precision should not be ``pushed too far'', for the processing done with the corpses used could possibly alter them from their ``wild'' state
			\item At first glance, Sabotta followed SOP and did good on eliminating subjectivity and thus achieving an ideal representation of a human body
			\item Is the ideal representation something that could be found? Or as a platonic ideal that could never be achieved but serves as a limit? Or does it only only express essential elements of what makes the bodies bodies?
			\item He bochup, only caring about eliminating interpretation
			\item But eventually even the process of amalgamating photos became seen as  a bit of subjectivity to be eliminated
		\end{itemize}
		\item Galton instead fully embraces the use of amalgamation in his drawings
		\begin{itemize}[label=\tiny$\blacksquare$]
			\item Use facial features to explain the tendencies of various groups of people
			\item But you need SOP to ensure objectivity
			\item Everyone will take pic, and the pics will be overlain to make one superpic
			\item But everything can be controlled, such as the amount of exposure time given to each individual pic, as long as it was scientific
			\item Even abstraction of features would be out of the artists’ control
			\item You can see the criminalness in a person by comparing his face to his ideal criminal science picture
		\end{itemize}
		\item What’s interesting is that even while scientists were restraining artists’ artistic tendencies, the scientists themselves also saw some sort of moral imperative in removing subjectivity arising from their own experimental processes
		\begin{itemize}[label=$\circ$]
			\item Astronomy man accounting for his own reaction speeds by adjusting results
		\end{itemize}
	\end{itemize}
	\item Scientists stop using artistic aids: instead switch to using multiple pics systematically curated to make your atlases
	\begin{itemize}[label=$\circ$]
		\item If you only use one stronk combined pic, how can you be sure what constitutes a deviation from the norm? That's subjective
		\item Grashey says use multiple pics and representations
		\item The multiples would ``evoke a class of patterns in the mind of the reader''
		\item Instead of just showing what the authors felt was the ideal sample, there is a shift towards offloading such a responsibility to the reader.
		\item The readers would be shown a bunch of unaltered, unfiltered samples and they would be the judge of what to do and how to interpret the information presented
		\item Instead of improving the visuals, you improve the reader's ability to discern
	\end{itemize}
	\item New photomechanical techniques however, did not remove suspected sources of subjectivity, only moved them somewhere else
	\begin{itemize}[label=$\circ$]
		\item The subjectivity is no longer in rendering, it's in the process of data gathering
		\item For example, in x-ray photography, the bias can come in the form of angles obscuring details or negligence on the part of the operator of the machinery
	\end{itemize}
	\item The x-ray appeared (in court at least) to displace all other forms of knowledge 
	\begin{itemize}[label=$\circ$]
		\item ``Usefulness and infallibility are not identical'' (pg32)
		\item Photographs were not absolutely accurate, but it had ideological significance in that people believed in the photographic dream of absolute reliability and objectivity
		\item Maybe cuz it looks better than some hand-made thing?
		\item Because of all the advances that photography has made, to challenge it in favour of the traditional method would be to eschew reliability in favour of carelessness and ineptitude
		\item ``We tend to trust the camera more than our own eyes''
	\end{itemize} 
	\newpage
	\item Another objection was much more radical: photographs should not be used as evidence at all
	\begin{itemize}[label=$\circ$]
		\item They show you what is captured, not what is true
		\item Use professional knowledge and personal discretion, don’t trust so much that you use it as definitive proof
		\item One response was to have witnesses to the production of the image, another to have a professional mediate the image and the public, another for the surgeons to elevate themselves to not require an intermediary interpreter
		\item Just abandon it entirely cuz knowledge and experience > eye
	\end{itemize} 
	\item While doctors could use experience and judgement as a weapon against falsehoods, scientists could not because they frequently explored the realm of the undocumented and unfamiliar
	\begin{itemize}[label=$\circ$]
		\item Instead of handling the equipment, get a technician to do it: removes confirmation bias 
		\item But at the same time, you cannot trust the technician not to fuk it up, so you monitor them while they took the photos
		\item Sacrifice accuracy for objectivity: maybe the photos suck at capturing colour?
		\item Leave the photos to speak for themselves, don’t touch them as much as possible unless to standardise them in some way
		\item NO TOUCHY
		\item Affected decisions about lab procedure and representational strategy
	\end{itemize}
	\item Instead of just photographs, pictorial representations are also a thing
	\begin{itemize}[label=$\circ$]
		\item Graphical method: universal in terms of language and discipline
		\item Anyone could read a graph
		\item THE WORDS OF NATURE
		\item If you anyhow draw correlations between stuff that shouldnt be correlated, the creation of shitty graphs can be an indicator that scientists are wasting their time in that direction
	\end{itemize}
	\item Technology help to make accurate and unbiased representations. Even if these representations lacked precision, objectivity rules
\end{itemize}

\subsubsection{Objectivity Moralized}
\begin{itemize}
	\item Atlas makers would rather present inaptly in their illustrations e.g. bad color, blurred boundaries, than to be suspected of introducing bias into their figures.
	\item Atlas makers' responsibility: presenting figures that would teach the reader how to recognize the typical, the ideal, the characteristic, the average or normal.
	\item \textit{Caught between the charybdis of interpretation and the scylla of irrelevance, atlas makers worked out a precarious comprise: They would no longer present a typical phenomena but rather a scatter of individual phenomena that would stake out the range of the normal.}
	\begin{itemize}[label=$\circ$]
		\item Doing so will help them achieve their responsibility of letting the user self-learn how to distinguish items
	\end{itemize}
	\item Mechanical objectivity thus pruned the ambitions of the atlas, but also transformed the ideal character of the atlas maker
	\item Knowledge and judgement are titles to authority and authorship; else any noob can also publish a scientific atlas
	\item Science demands self-discipline <insert random scientists name like Faraday>
	\item Scientists can assume priestly functions in a secularized society with the professional ethos of noninterventionist objectivity
	\item Objectivity is a morality of prohibitions rather than exhortations
	\begin{itemize}[label=$\circ$]
		\item Say no to projection \& anthropomorphism
		\item Say no to the presence of hopes and fears in images and facts
		\item Say no to anything that introduce subjectivity \& interpretation
	\end{itemize}
	\item Aperspectival Objectivity: Simplify by conflation
\end{itemize}

\subsection{Notes on Class Recitals}
\subsubsection{Objectivity throughout the years}
\begin{itemize}
	\item Relatively new, only a few hundred years
	\item They are not transcendental, the idea of objectivity transforms over time
\end{itemize}
\subsubsection{Mechanical objectivity}
\begin{itemize}
	\item Emerged in the late 19th century 
	\item A timeline of what an ideal scientific depiction of stuff should look like:
	\begin{itemize}[label=$\circ$]
		\item 1500s: HUNTING HORN VS SEAMONSTER DOOT DOOT (maps)
		\item 1700s: remove blemishes and an attempt at accuracy. Creating an ideal by touching up the actual thing (Albinus)
		\item Later: Draw them as they are, and let the pic represent a group or class of things
		\item 19th century: mechanical objectivity
		\item Categories used to describe the shift in depictions over time:
		\begin{itemize}[label=\tiny$\blacksquare$]
			\item Typical - one image or object that represents the group of that object
			\item Ideal - perfect PAAFUEKUTO but might not be real
			\item Typus - common characteristics of your samples and smoosh all of them together: what's in all the oranges you observe? Flies FRIES?
			\item Characteristic - actual existing sample, let it represent the group of stuff the sample is from: Orange A is representative of oranges, and any features on this orange will be thought of to exist on all other ``normal'' oranges
		\end{itemize}
	\end{itemize}
	\item Technological determinism
	\begin{itemize}[label=$\circ$]
		\item Some might say that technology drives the quest for objectivity because TECHNOLOGY HAS USHERED IN A NEW ERA OF ACCURATE RECORDING
		\item But actually,  objectivity has always been around, just that technology has enabled it
	\end{itemize}
	\item Epistemological history
	\begin{itemize}[label=$\circ$]
		\item Science and art today are considered to be disparate
		\item But this was not so in the past
	\end{itemize}
	\newpage
	\item Atlas images
	\begin{itemize}[label=$\circ$]
		\item What are the common characteristics of an example of a typus? The choice of what goes into a typus is not necessarily inaccurate, but it is certainly subjective
		\item A shift in responsibility from the artist to the viewers to decide if something is representative or true
	\end{itemize}
	\item PERSPECTIVE
	\begin{itemize}[label=$\circ$]
		\item Rules for proportions to limit how drawings are made
		\item OPTICAL CONSISTENCY \textasciitilde\ Latour
		\item Camera Lucida: reflects light from the object onto a piece of paper so you can trace the image and make ACCURACY WOW
		\item Camera Obscura: pinhole and room, light from outside shines on wall and paint over
	\end{itemize}
	\item THE INVENTION OF THE CAMERA
	\begin{itemize}[label=$\circ$]
		\item TRACING SUCKS THERES PEPLE INVOLVED. WE MAKE SUPER DUPER CAPTURE DEVICE
		\item Henry Fox Talbot and the calotype method
		\item The SUN'S ARTISTRY. OBJECT FORMS MATHEMATICALLY PRESERVED. LEGITERLLY AN EMANATION OF THE REFERENT. THE LIGHT LICKS MY EYEBALLS STRAIGHT FROM THE SOURCE OBJECT
		\item UNIVERSAL LANGUAGE. Pre-linguistic. CAVEMEN BUT STRONGER
		\item MORE CREDIBILITY
	\end{itemize}
	\item Composite photographs
	\begin{itemize}[label=$\circ$]
		\item Overlay transparencies of criminal mugshots as a filter and make a gattai picture
		\item Francis Galton
		\item A typus of criminal mugs
		\item ELIMINATING HIS OWN INFLUENCE OVER THE PICTURE MAKES IT OBJECTIVE WOW THE MACHINE IS DOING IT FOR ME
		\item Still doing this composite criminal photo shit, the only difference being the tech is smarter now
		\item Have you watched minority report?
	\end{itemize}
	\item Visualisation tech makes stuff, but what are we supposed to make of it?
\end{itemize}

\subsubsection{Prof Samson's thoughts}
\textbf{Argument:}\quad The two authors argue that the modern notion of objectivity, which they call ``mechanical objectivity'' (p. 82), emerges in the late 19th century in Europe. (Their study is limited to European contexts.) Prior to this period, scientists and illustrators also thought they were representing nature `accurately' but their notions of what `accurate' was not the same as ours now.

\bigskip

\noindent\textbf{Method:}\quad They look at old atlases from roughly the 16th century to the 19th century and from the images and text of these primary sources, they interpret a history of the notion of `objectivity'.

\bigskip

\noindent\textbf{Mechanical Objectivity:}\quad This is the term they give to the modern notion of objectivity. It refers to the current desire to produce images that capture nature visually with as little human interference as possible. That is, they feel that now scientists want to show objects as they are, complete with blemishes and other details. They argue that scientists today feel that the exercise of judgment on matters of aesthetics distorts or makes an image less true to nature.

\bigskip

\noindent\textbf{Ideal Image:}\quad In the past, however, there were different ideas as to what an image should look like. One idea that influenced the production of scientific representations is the notion of the `ideal.' Ideals are images that represent an idea of a perfect specimen of a larger group of things. For example, someone may feel that the `perfect man' is 300 meters in height, has purple hair, and rainbow eyes. (Of course, this is not the case, but I exaggerate here to help me explain the concept.) They would then make their representation accordingly, as they feel the subject should look. This ideal does not necessarily exist in reality but is sort of a measuring stick against which other instances can be compared.

\bigskip

\noindent\textbf{Typical Image:}\quad A typical image, or typus, is one that combines a number of common characteristics of an object. These images are based on observation of many examples. The scientist or artist then chooses (makes a judgment) as to what characteristics of the class of objects is `typical' and creates an image that shows those characteristics. So if we take a zebra as an example, the scientist would look at many zebras and then decide what characteristics of the species makes it distinct: white coat with black stripes, a vertical mane of hair on the neck, a tail, four hooves, etc. They will then draw a zebra with the characteristics they feel are representative of the animal as a group/species. The image they draw does not exist in reality.

\bigskip

\noindent\textbf{Characteristic Image:}\quad A characteristic image is the image of a single, existing example of a group of objects as it is. So, for example, if you want to depict what a salmon fish looks like, you would pick one from the wild, take a picture of it or draw it as it is, and then say that represents all the animals in the species. That is, the one stands for all. In this case, the judgment takes place during the choosing process, though her examples show that it also takes place in the rendering process.
\begin{itemize}
	\item Each type involves human judgment and this was seen as okay prior to the late nineteenth century. In fact, they argue that the expert was expected to use his or her skill and expertise to make decisions about what a thing should look like. This changes later in time so that the judgment is then seen as distortion. They do not really discuss why this change happens.
\end{itemize}
\noindent Other key points that come up in their essay include `representation as a moral exercise' and `responsibility of interpretation'.

\newpage
\section{W6: Maps}
\subsection*{Weekly Readings}
\begin{itemize}
	\item Thongchai Winichakul, \textit{Siam Mapped} (Honolulu: University of Hawaii Press, 1998),
	Chapter 6, pp. 113-127.
\end{itemize}

\subsection{Notes on Readings}
\subsubsection{Siam in Western Maps}
The Chinese made maps of the Malay peninsula and European adventurers often used these in their explorations. But Siam (Thailand) only appeared on maps relatively late.
\begin{itemize}
	\item French and Dutch in the 17th century very good at making maps and exploring the Orient
	\item French envoys and cartographers of Louis XIV published maps of Siam and passed on their knowledge to other EU countries
	\item But only the outside lol. The insides of Siam were still wtf until the 19th century
	\item Very little or even inaccurate info of what was in Siam was available until then
\end{itemize}

\subsubsection{Mercator Projection}
\begin{itemize}
	\item Many squares to be filled in lol
	\item Modern mapmaking as the impetus for exploration and documentation
	\item Exploration and documentation motivated by COLONIALISM
	\item Capt. James Low and his anyhow made map says a lot about European knowledge of Siamese geography in the 1st half of the 19th century
	\item They take their info from the natives, who had little knowledge on the actual geography
	\item The anyhow made map all look quite different but they have striking similarities in the location of Siam in the Chao Phraya valley and upper part of the Malay Peninsula. On the Eastern border, Siam was blocked off by mountain ranges, separating it from Laos and Cambodia
	\item While the Europeans did acknowledge the constantly shifting boundaries of what constituted Siam, they were thinking of formal occupations of land by more powerful nations. Instead, what really happened was \textbf{simultaneous control of territory} by multiple parties without clear delineation of boundaries
	\item The maps the Europeans made were a reflection of what they say Siam as, as well as what the locals see what constituted Siam and what did not
	\begin{itemize}[label=$\circ$]
		\item The locals did not see Lanna, Lao and Cambodian regions as part of Siam
	\end{itemize}
\end{itemize}

\subsubsection{Western Mapping in Siam}
\begin{itemize}
	\item The Siamese themselves were not so interested in making maps either until King Mongkut
	\item Except war routes to Bangkok
	\item British messenger who wanted to travel by land from Bangkok was misled by a Siamese official to prevent knowledge of land routes from Bangkok to British occupied territories
	\item Under Mongkut, the elites became familiar with Western cartographic tools and techniques
	\begin{itemize}[label=$\circ$]
		\item Maps were often gifted to Mongkut: important to him
		\item The elites saw these maps and might have even visited the countries depicted by said maps. Why not put Siam there too? SHOW THE WORLD OUR SPLENDOR
		\item But even to the Siamese elites, maps of Siam were largely absent
		\item But these \textbf{elites were a lot more welcome to the idea of map making}, actively encouraging and INNOVATING it
		\item But little effort had gone towards mapping the inside of Siam, until some French guy did it and the elites said ``dude we gotta do it too''
		\item Map and geography important because infrastructure had to be built, and you can’t build buildings without knowledge of where you are building
		\begin{itemize}[label=\tiny$\blacksquare$]
			\item Possible Chicken and egg problem?
		\end{itemize}
		\item Need knowledge, technicians and facilities, but because they were new to this, the majority of work was still done by foreigners
		\item Henry Alabaster: although he was not a cartographer, he was an engineer and he plotted some nice telegraph lines for Bangkok. Also Auguste Pavie.
		\item In 1880, British India government requested permission from the Siam court to place posts in Siamese territory to aid their \textbf{triangulation}. Their proposed marker sites were dangerously close to vital locations to the Siamese so the officials were spooked
		\item They had a meeting because this sort of behaviour looked like the makings of an invasion, which was a reasonable response because the construction was not for building buildings in a particular locality. Maybe they didn’t understand what triangulation was?
		\item While there had been topological surveys before, none of them were of this large a scale and none of them covered the capital 
		\item The marking points were also sacred. So insensitive D:<!
		\item Employ James Fitzroy McCarthy to do the map of Siam for the Siamese government to connect with British triangulations to allay the fears of the Siamese
		\item Triangulations finished, surveys for the map of Siam \\ $\rightarrow$ This \textbf{allowed for modernising projects}
		\item But technology not welcome by the locals because they \textit{feared confiscation of property}. Obstructions happened and surveyors were observed at every step
		\item They even killed a dude for trying to survey sad boi
		\item ``Surveying was regarded as of no use in the administration of the country''
		\item ``Far more likely to serve the purposes of a future invader''
		\item But there was change and mapping had become as necessary to Siam as other infrastructural advancements
		\item The Siamese had mapping officials, trained by the British. There were also Western style schools that taught math, astronomy and the use of Western devices, also topography and coordinates
		\item Not much was recorded in history about this school, maybe because it wasn't general, but educated based on a particular job
		\item Royal survey department for mapping by the government established
	\end{itemize}
\end{itemize}

\subsubsection{The Making of ``Our'' Space by Maps}
\begin{itemize}
	\item Phraya Maha-ammat (Interior Ministry 1892) claimed that his superiors knew the towns by name but cannot identify them on the map, so maybe they don't actually care about the border problems
	\begin{itemize}[label=$\circ$]
		\item Bad assumption that highlights \textbf{the transition from the importance of knowing places purely by names to knowing places by maps.}
	\end{itemize}
	\item Triangulation increased needs for mapping rapidly
	\begin{itemize}[label=$\circ$]
		\item Small challenge: Construction
		\item Big Challenge: Provincial Administration
	\end{itemize}
	\item However, while it is easy to identify towns by name, domains owned by lords are harder to be identified,  as it could be discontinuous. Even the court did not know its realm territorially.
	\item Mapping is a cognitive paradigm and practical means of a new administration
	\begin{itemize}[label=$\circ$]
		\item Assisted in the shift from traditional hierarchical relationships of rulers to the new administration on a territorial basis
		\item Chulalongkorn believes that the reorganization and mapping of the border can be a way to counter the French over the Mekhong region. Hence expeditions led by Bangkok troops begin to have the accompany of mapping officials.
	\end{itemize}
\end{itemize}

\subsubsection{Mapping Cross Fire: A Lethal Weapon Unleashed}
\begin{itemize}
	\item Maps were used as a basis for negotiation for claiming territory. The most prominent one is the 1887 McCarthy Map, that is recognized as the first modern map of Siam.
	\item The lack of boundary lines in Pavie's Maps was not certain to be deliberate for leaving the question open for additional claims, or due to his scientific rigidity which did not allow him to specify a boundary line while the demarcation was not yet finished
	\item Mapping helps to spearhead conquests. Force defined the space. Mapping vindicated it. Without force, mapping alone inadequate to claim a legitimate space. But the legitimation of the military presence was always substantiated by map. 
	\begin{itemize}[label=$\circ$]
		\item James F. McCathy knows the area best so he's involved in a lot of strategic planning, leading to successful military campaign in 1885-1888, as he could recommend appropriate specific towns for occupation.
	\end{itemize}
\end{itemize}

\subsection{Notes on Class Recitals}
\begin{itemize}
	\item Some bg info:
	\begin{itemize}[label=$\circ$]
		\item Some thai dynasties, Bangkok is just the most recent one
	\end{itemize}
	\item How did these kingdoms think about territories?
	\begin{itemize}[label=$\circ$]
		\item \textbf{Power, instead of coming from control of geography, comes from control over numbers} (of people)
		\item Rulers had control over the people in their immediate vicinity and these rulers in turn would send tribute to a bigger city
		\item Some smaller cities might end up sending tributes to multiple bigger cities due to geography 
		\item Some of the people in these smaller cities started to grow tired of the monarchy and wants a different standard of government
		\item Revolution and shit, instituted a constitutional monarchy. Is like monarchy but with parliament. Got a few coups, with the constitution changing each time
		\item Chill out bruh so much fighting smh, so much political violence
		\item \textbf{Thongchai} was one of them student protesters in the 1976 protests. 
		\begin{itemize}[label=\tiny$\blacksquare$]
			\item Thongchai wanted to explain why and how this violence happens, why people shoot each other ouch
			\item Part of his answer involves the map
		\end{itemize}
	\end{itemize}
	\item Siam was quite against the creation of maps, thinking that maps were of no use in administration, only good for foreign invasion
	\begin{itemize}[label=$\circ$]
		\item There was only demand for mapping after the development of the Mercator projection and the recognition that there was a need to fill in that empty square
		\item Prior to formal mapping efforts, evidence and recounts were the only sources of information on the geography which were very unreliable and sometimes even intentionally misleading
		\item King Mongkut was more interested in this and wanted to open Siam to the world
		\item He wanted more formal borders for use in administration. No ambiguity in sovereignty and recognition from others over alleged territorial borders (Thesaphiban)
		\item Also infrastructure
		\item New administration on a territorial basis, people had to register their houses
		\item If they fought with weapons they would lose against the French. They needed to negotiate and gain recognition for their own borders. That's what mapping is for
	\end{itemize}
	\newpage
	\item Map race wars
	\begin{itemize}[label=$\circ$]
		\item Fight the French to finish maps faster
		\item 1888 Thaeng confrontation
		\begin{itemize}[label=\tiny$\blacksquare$]
			\item Both parties wanted to map, but they also did not know want the other to map
			\item Go back to Bangkok to negotiate which map should be accepted
			\item They negotiated using maps to see who deserves the credit and which map was the one to use
			\item The more proof you have, the more likely it is to be accepted because you have more allies (Latour says hi)
			\item The French were colonising the area and says WE CONTROL THE MEKHONG. But the Siam says NO and hired McCarthy
			\item Then MLG \textbf{competitive mapping} to fight for legitimacy of their claims
		\end{itemize}
	\end{itemize}
	\item Maps and sovereignty
	\begin{itemize}[label=$\circ$]
		\item You can't just map out borders without the military power to back up your territorial claims
		\item Power lets you control the space, maps let you declare to the world THIS SHIT IS MINE NO TOUCH
		\item Before the mapping, the Siam kingdom had no boundary, only approximations, kingdoms with vague levels of influence
		\item The Siam boundary only came into existence around 1888, it did not exist prior. Same for Cambodia and Laos
		\item In the readings from previous weeks, mapping was just a way of representing what exists physically
		\item Here, in drawing boundaries, there is CREATION. You are creating borders. Borders are not natural, they are man-made.
		\item Formalises agreements and disagreements about territorial control
		\item Thongchai: Thailand was created in the process of mapping. Before mapping, there was no Thailand that the world recognises and knows, just a blurred mass of territories. Only in 1888 did modern Thailand formally exist. Historiographical argument about Thailand’s existence. In the mapping, the geobody is created
		\item The Thais had power levels like dragonball Z
		\begin{itemize}[label=\tiny$\blacksquare$]
			\item Society was organised around the Sakdina rank
			\item If you were a scrub peasant you would be beholden to the nobility in charge of you so you had to give tribute
			\item Power was not territorial, but more of ``these people owe you stuff'' even though there was a vague sense of territory
		\end{itemize}
	\end{itemize}
	\newpage
	\item Why is the monarchy so sensitive? Why they arrest and violence?
	\begin{itemize}[label=$\circ$]
		\item The geobody is seen as an extension of the king
		\item The British want some states from Burma 
		\item ``Wake up, Thai People'' poster $\rightarrow$ The communist wants to ``eat up'' the state, so depict the guy with wide mouth and legs over the geomap
		\item Attacking these offenders is seen as patriotic and necessary to protect the kingdom
		\item Territories like Patani that were absorbed into Thailand almost arbitrarily through the mapping in 1880’s might not actually feel like they belong in Thailand
		\item FIGHT AND MURDER THEM TO PROTECC TERRITORY
		\item LAY THEIR BODIES TO FORM OUR GEOBODY
	\end{itemize}
	\item Characteristics of A Nation (State)
	\begin{itemize}[label=$\circ$]
		\item Exclusive (discrete) in terms of Citizenship \& Territory 
		\item Culture - language, beliefs and ideas shared by a majority of citizens, history
		\item Identity 
		\begin{itemize}[label=\tiny$\blacksquare$]
			\item A nation is mostly commonly used to designate larger groups/collectives of people with common characteristics attributed to them, including language, tradition, customs, habits and ethnicity
			\item An entity through which people in the world are organized that is based on fixed territorial boundaries
		\end{itemize}
		\item The nation state is a relatively new concept - mid 19th century
	\end{itemize}
	\item Before the Nation State
	\begin{itemize}[label=$\circ$]
		\item How do people think of themselves? 
		\begin{itemize}[label=\tiny$\blacksquare$]
			\item Subjects of a leader/king.
			\item Followers of Religion
			\item Heritage and ethnicity, dialect
		\end{itemize}
		\item \textit{Relevant Source: Benedict Anderson - Imagined Communities}
		\item They did not necessarily identify themselves as members of a nation
	\end{itemize}
	\item Thailand violence happens because people started identifying themselves with a nation state and they see it as something to be fought for, to be protected. When this sovereignty is threatened, people turn to violence to fight for it
\end{itemize}

\newpage
\subsubsection{Prof Samson's thoughts}
Thonghai makes two main points. The first is a historical argument about Thailand. The second is a theoretical claim about nation-states.

\medskip
\noindent\textbf{Historical Argument}
\begin{itemize}
	\item The Kingdom of Siam during the early to middle of the 19th century was based in Bangkok. The rulers in Bangkok had the allegiance of smaller cities and towns in the Chao Phraya basin and to the south of Bangkok. The rulers in those cities and towns paid tribute in form of goods, money and people (as laborers, soldiers, etc.). There were no hard and fixed boundaries. In fact, much of what is now Thailand were separate kingdoms. 
	\item It is only in the late 19th century that the modern day boundaries of the country were created through mapping. 
	\item In part, the Kings of Thailand started to map the region in order to counter French claims to territory along Mekhong. In other words, the Thai kings needed maps (immutable mobiles) to create a network of allies, or `facts', to present to the French colonial forces that they did indeed `own' the territory being disputed.
	\item In the process of mapping the kingdom, the `geo-body' (or territory) of present-day nation-state was created. The Thailand that exists today was not some ancient entity with a long history, but a modern invention that came about through a visualization practice - mapping. 
\end{itemize}
\noindent\textbf{Theoretical Claim}
\begin{itemize}
	\item Maps (and other visual, material practices and objects) produce the territory of a nation-state, making it seem real. The `geo-body' of the nation then can be projected back into time and taken as something that has existed for a long time. This provides people with a sense of continuity and belonging that contributes to feelings of nationalism. 
	\item In other words, the nation-state as we know it today is in part a creation of visual practices.
\end{itemize}
\noindent\textbf{General Notes}
\begin{itemize}
	\item The chapter can be read a number of ways in relation to earlier readings for the class. On page 116, for example, there is a bit where Thongchai says that Europeans knew a lot about Siam because they had already seen the Kingdom on a map. This, hopefully, reminds you of the idea of immutable mobiles from week 2 in which Latour talks about how things like maps can transport information and thus lead to the ability to ``dominate from a distance''.
	\item You can think also about why mapping came to be adopted. What were the purposes? A few come up in the chapter:
	\begin{itemize}[label=$\circ$]
		\item Taxes
		\item Land ownership
		\item Border disputes
		\item Colonial land acquisition
	\end{itemize}
	\item Some might see this as evidence that mapping is not a value neutral exercise. That is, people might not necessarily make maps just because they want to know something about geography. They map because they have political or economic motives. The `competitive' mapping between the French and the Thais shows this as well.
\end{itemize}

\newpage
\section{W9: Clocks}
\subsection*{Weekly Readings}
\begin{itemize}
	\item Peter Galison, \textit{Poincare’s Maps} (New York: Norton, 2004), Chapter 5, pp. 221-293.
\end{itemize}

\subsection{Notes on Readings}
\subsubsection{Einstein's Clocks}
\textbf{Materialising time}
\begin{itemize}
	\item Rails and telegraph arrived in Switzerland, came also the movement to synchronise clocks, due to national pride and economic need
	\item Matthaus Hipp banned from Wurttemburg, went to Switzerland, makes good timepieces
	\item Many timepieces, need sync with repeatability, make many patents for sync systems
	\item Before there was one standard for time, shit was messy af, unlike in Switzerland where everything was standardised.
	\item Time keeping was economic necessity and cultural imaginary in Switzerland
	\item Albert Farvager complained not enough resources were allocated to helping timekeeping technology keep up with other tech because timekeeping very important
	\item Use electrical instead of mechanical clocks cuz uniformity and universality
	\item Imagine train traffic control without proper timekeeping oh no time anarchy
	\item Time anarchism brought about something bigger: disintegration of personal and societal regularity
	\item Only electric clocks could unify all standards of timing and provide the world with one single time
	\item Political, profitable and pragmatic
	\item Farvager proposed a central electric giant clock tower that would send signals so that all other clocks would be synced to it
\end{itemize}

\medskip

\noindent\textbf{Theory-machines}
\begin{itemize}
	\item Poincare's polytechnic emphasised more on the beauty of theory while the ETH focused more on practical engineering and application
	\item Ether is fake and gay, modern EM theory is the more widely accepted one nowadays
	\item Einstein wanted to know about electrodynamics for moving bodies
	\item With success comes intellectual succ: cross referencing other eminent intellectuals who may or may not be correct to defend your own theories against objections
	\item Dude found a job at the patent office. Nice going m8!
	\item Dude's spicy critique found acceptance from his old thesis evaluator
	\item He found some students for math and physics tutoring, and they did the cool informal gatherings
	\item The gatherings talked about a guy called Mach, whose ideas about the futility of Newtonian absolute time and space caught on with Einstein
	\item They also discussed Karl Pearson - instead of the synchronised time established above as an absolute, it derives its timing from the rotation of the Earth. But this rotation isn’t always regular, it can be influenced
	\item Also Hume's Treatise on Human Nature
	\item They read and like Poincare's Science and Hypothesis
	\item A long line of philosophers, physicists and mathematicians asserting that our conception of measured time was just motion relative to other motion
	\item We wouldnt have Einstein's theory of relativity if not for these works: ``Einstein's advocacy for a  procedural notion of simultaneity against metaphysical, absolute time'' was a direct consequence of the desire to improve physical knowledge carried out by his predecessors
	\item Principles over model building and instead of principles being definitions used for their convenience as with Poincare, principles supported physics and science
\end{itemize}

\medskip

\noindent\textbf{Patent Truths}
\begin{itemize}
	\item Patent office job good for critical thinking, trained Einstein to avoid falling into the trap of being swept along by the inventor’s way of thinking and approving something stupid
	\item AKA dont believe that all assumptions hold true, test them and shit
	\item ``As a physicist you dont know shit about technical drawings so until you learn i cant make you permanent staff''. In time, he managed to learn the VISUAL LANGUAGE OF THE PATENT WORLD and managed to get promoted and stuff
	\item Patents on electric time were many
	\item Everyone was doing it, but the Swiss were the gayest of them all
	\item Gyrocompass served as a model for Einstein’s theory of the magnetic atom
	\item Reading and writing like attacking and critiquing a patent, very strict and high standards
	\item Many time coordination patents, inspired him to make the link between time and electromagnetism
	\item If we want to describe motion, we provide spatial coordinates and time. Without the ether as a point of reference for absolute time, any point of reference is as valid as any other
\end{itemize}

\subsubsection{Republican Calendar}
\textbf{Modern Western Standard Temporal Reference Framework}
\begin{itemize}
	\item Social organizations presupposes a \textbf{coordinative calibration} of subjective temporal references 
	\begin{itemize}[label=$\circ$]
		\item Impersonal, collective/intersubjective modality of temporal reference
	\end{itemize}
	\item Framework is based on relatively universal relevance and combined use of the following
	\begin{itemize}[label=$\circ$]
		\item Gregorian Calendar
		\item Christian Era
		\item International Standard Time 
		\item Clock Time
	\end{itemize}
	\item Social Artefacts
	\begin{itemize}[label=$\circ$]
		\item \textbf{A standard system of units of time:} A Year 12 Month, 7 mths 31 days, 4 mths 30 days, and 1 is 28 days (if not leap year)
		\item \textbf{It is merely conventional} that our era begins with the birth of Jesus Christ, our year on Jan 1, our week on Sunday and our day at midnight.
		\item \textbf{Standard time-reckoning and dating frameworks}
	\end{itemize}
	\item Durkheim critique Kant's apriorist theory of knowledge
	\item Even within one and the same society, the relative validity of the framework has no absolute validity. Still depends on entirely arbitrary social conventions.
\end{itemize}

\medskip

\noindent\textbf{The Reform and Its Social Meaning}
\begin{enumerate}
	\item The Committee of Public Infrastructure proposed a reform of the French Republican Calendar, and was put into effect by the National Convention on November 24, 1793.
	\begin{enumerate}[label=\alph*.]
		\item \textit{New Beginning:}
		\begin{enumerate}[label=\roman*.]
			\item Christian Era replaced by Republican Era, which began on September 22, 1792, the day on which the French Republic was founded
		\end{enumerate} 
		\item \textit{New Annual Cycle:}
		\begin{enumerate}[label=\roman*.]
			\item Jan 1 replaced by September 22 as New Year's Day.
		\end{enumerate} 
		\item \textit{Uniformization of the months:}
		\begin{enumerate}[label=\roman*.]
			\item Every month is 30-day. 
			\item The Five Complementary Days \textit{Sansculottides}, were grouped together at the end of the year, like the ancient Egyptian calendar.
			\item A sixth intercalary day was added on leap years (“\textit{sextiles}”), on the 3rd instead of the 4th yr of every group of 4 years (“\textit{Franciade}”)
		\end{enumerate}
		\item \textit{Abolition of the seven-day week and Sunday}
		\begin{enumerate}[label=\roman*.]
			\item Each 30-day month is divided into 3 10-days cycles (“\textit{decades}”).
			\item Sunday replaced by “\textit{Decadi}”, which is only celebrated every 10 days, as the official day of rest. Only 1 rest day for every 10 days spent hmm...
		\end{enumerate} 
		\newpage
		\item \textit{Decimal Subdivision of the day}
		\begin{enumerate}[label=\roman*.]
			\item Days are 10 decimal hrs, wait, so i can end up sleeping an actual entire day if i’m in France oh wow 
			\item Hours are 100 decimal minutes
			\item Minutes are 100 decimal seconds
		\end{enumerate}
		\item \textit{New Nomenclature:} Within each new \textit{concept}, each day and month within the new calendar is also renamed
		\begin{enumerate}[label=\roman*.]
			\item Months are named after seasonal aspects of nature
			\item Special \textit{Sansculottides} days are named after Values.
			\item Monday-Sunday renamed after trees, plants, seeds, roots, flowers, fruits, farming implements and animals.
		\end{enumerate}
	\end{enumerate}
	\item The Architects of this calendar hope to gain social control by imposing a new rhythm of collective life, but also bring about a \textit{total symbolic transformation} of the temporal reference framework.
	\item This calendar is introduced during an age which advocated total obliteration of the old order in the name of progress and modernity.
	\item The Republican Era marked the total discontinuity between past and present.
	\item Re-adopted by the Paris Commune in 1871.
\end{enumerate}

\medskip

\noindent\textbf{Secularization}
\begin{itemize}
	\item Calendars always closely associated with religion, and thus entirely controlled by priesthood.
	\item Pre-Republican France's calendar is introduced by Pope Gregory XIII, thus symbolically associated with Catholic Church.
	\item Spain and Portugal adopt in 1582, Greece and Russia adopt in 1974
	\item Sylvain Marechal simply proposed to change some parts of the calendar in 1788, but he got his house destroyed for trying to destroy religion. 5 years ltr he had a new house with a big warning saying the calendar of French Republic must not be as retarded as the Roman Church
	\item The Church's control of the temporal regulation of social life was challenged by the Revolutionary Calendar that abolished the seven-day week and replacement of Sunday. The ten-day rhythm also disrupted church-attending practices. People have to work longer before rest day.
	\item Major Characteristic of any calendar is that it interrupts the continuous flow of time by introducing some regularly recurrent critical dates and the events that constitute these temporal reference points usually have a particular symbolic significance in the society.
	\item Napoleon managed to restore the Gregorian Calendar 
\end{itemize}

\medskip

\noindent\textbf{Naturalisation}
\begin{itemize}
	\item A purely physico-mathematical analysis of the new French calendar, which would ignore the nomenclature innovations of the poet Fabre d'Eglantine would never be enough to tell us about its social meaning
	\item Names have meaning and calendrical systems should be seen as the product of a relationship between intervals of time and symbolic frameworks of meaning
\end{itemize}

\subsection{Notes on Class Recitals}
\subsubsection{Clocks}
\begin{itemize}
	\item Clocks: How Einstein come up with relativity lol? 
	\item 1861: people talking about synchronising clocks, one standardised reading of time
	\item Time anarchy was a bitch: different places had different standards of time. One country, 200+ timezones
	\item Use pikachu, the prevailing theory about EM was the ether, basically the medium for the perturbations in EM fields through which EM waves propagated
	\item Scientific complacency on the part of the physics community and Einstein, for his contrarian views was mocked and stuff
	\item Dude got a job in a patent office and was able to make a kool kidz klub where he and his friends discussed anything they found interesting
	\item Discussions found that there was no such thing as absolute time: the unit of time used back then was based on the Earth's rotation, but this rotation isn’t a constant, and so should NOT be used as such
	\item Eventually found his way back into academia by writing papers - speaking the language of the network!
	\item He saw lots of patents: time synchronisation networks mockups, clock designs and actual clocks, he had visual aids.
	\item FUcK 1800s simultaneity: you can't really synchronise clocks absolutely accurately because signals need time to travel as well
	\item There is an inseparable relationship between time and signal velocity
	\item Einstein says there is no absolute notion of time, but time is dependent on reference frame
	\item YOu caNnoT Grasp the TRUe FoRm Of TimE! 
	\item Can be characterised by a reliable, periodic, measurable physical phenomenon
	\item In the early 1990s, France was experiencing geopolitical worries: Superior British cable system, colonial uprisings, conflicts with Britain
	\item French pride and the Eiffel tower: use Eiffel tower as the standard for clocks everywhere. Some other ppl were like fuck dat. The Americans make quartz clocks and GPS, making the above obsolete 
	\begin{itemize}[label=$\circ$]
		\item How about the Joss stick of time uwu Clocks are not the only mediums for representing/ visualizing time
		\item Joss stick is a form of \textbf{task-based time}
	\end{itemize}
	\newpage
	\item The nature of creativity and innovation:
	\begin{itemize}[label=$\circ$]
		\item Not just about spontaneous genius
		\item Often inspired by the mundane, born from rumination
		\item You start recording stuff down so you can circulate ideas and discuss
		\item Physical, tangible, based on materiality, stuff you apprehend with your senses
		\item A relationship between the material and the abstract
		\item Praxis - putting theory into Practical Use
	\end{itemize}
	\item Time is a social construct, as the society needs to keep track of time to function: time sensitive applications such as train schedules. Before time synchronisation, the concept of a schedule was practically nonexistent. Productivity made time synchronisation necessary. War also made standardised time necessary. War makes many things
	\begin{itemize}[label=$\circ$]
		\item The July revolution: people, feeling oppressed by regular work hours, decided to start their revolution by shooting this bigass clock, a symbol of the source of their oppression
	\end{itemize}
	\item Timekeeping is an immutable mobile
	\item The American dream is ruined by a trumpet
	\item How much does it cost for you to buy some time: Too much
	\item Why need clocks when you have roosters that will wake u on time 7am everyday
	\item Timezones and national identity
	\item Other forms of time:
	\begin{itemize}[label=$\circ$]
		\item Task-based
		\item Seasonal
	\end{itemize}
\end{itemize}

\subsubsection{Zeruvabel}
\begin{itemize}
	\item Time as a reference framework in which to place events, conventions with which to define time - something that is ultimately intangible directly, but can be felt
	\item Zeruvabel argues that social conventions dictate changes to this framework. Calendars were a social construct and the traditional calendar was not unalterable: GUNDAM 
	\item French revolutionary calendar attempts to make such a change, in which its architects try to destroy the old framework and gain social control by influencing the tempo of daily activities
	\item Because it attempted to do away with church days, it challenged the church's influence
	\item The calendrical reform implied that the French revolution was more important than the birth of Christ, which ofc was met with objection
	\item Obviously it failed cuz we dont use this calendar lol
	\item One of the biggest reasons for failure was that there was no way to reconcile the old calendar with this new one and great efforts would have to be made to accommodate this change. The primary reason for the new calendar was a desire for secularisation and a display of revolutionary power but many found this reasoning weak. This, together with the fact that many people were still religious meant that there were opponents to the adoption of this French calendar
	\item We go back to Ol' Greg's calendar
	\item So inconvenient! Just like dvorak. 
	\item Rendering time allow us to dominate it, calendar can be a docile object / immutable mobile
\end{itemize}

\subsubsection{Prof Samson's thoughts}
\textbf{Time Anarchy and Synchronized Time}\\
Before the 1880s, most countries did not have a single, standard time. In fact, most countries had several different time zones, even for cities and towns very near each other. Galison calls this `time anarchy.' It was only around the 1880s that governments started to try to synchronize time in different cities and towns. From this, Galison tells us through examples that time is in some ways a social creation. That is, people create ways to think about and measure time. Time (as we know it now in terms of minutes, seconds, etc.) is not `natural'. It is human-made. This is a fairly radical idea in that it could lead one to think that there is no `time' `out there' in the world. So another way to think about it, if you prefer, is that there might be a `time' out there, but that humans simply make different ways of calculating it.


\bigskip

\noindent\textbf{Interests and motivations}\\
One question that might come to mind after reading this, is why would people want to synchronize time? What interests or motivations are involved in doing so? As we talked about in class, these can be many, like making sure trains do not crash, that you can know when your loans are due, when to attack an enemy in a war, and so on. In other words, the motivations can be related to production, business, military purposes, communications, and transportation. They can also be symbolic, in that a nation may want to have a single `national time'. This is important to consider because as a social scientist, then you can see that one way to study the world is to look at material objects (in this class, mostly images) and try to see what the design of those objects or images tell you about the society that produces them.

\bigskip

\noindent\textbf{What drives invention?}\\
A related question is ``What drives invention?'' Is it just the quest for new knowledge? Can knowledge be separated from the motivations that drive research (for example, military motives or financial profit motives)? Why were electric clocks not funded at first, according to Galison? Why were the best engineers in communications and power sectors? These questions are the sort that a social scientist asks and tries to answer because they will tell us something about how the world works.

\bigskip

\noindent\textbf{Materiality}\\
On the topic of creative processes, there is an idea that people think through things and through physical activity. That is, people don't necessarily just sit in a room and creative ideas just pop into their heads. Instead, people externalize ideas onto paper or in objects and then they work with those objects or drawings to create new ideas. Doing is thinking. In the case of Einstein, we can see from Galison's chapter that he thought about relativity through visualizations of time in the patent office. He saw and evaluated images of time networks and through his experience there, he was able to refine his ideas about time and motion. The creative process is for many people a physical activity, not a detached one where you just think.

\bigskip

\newpage
\noindent\textbf{Zeruvabel on The French Republican Calendar}
\begin{itemize}
	\item Zerubavel's article gives us an example of how political or other ideals, interests, or motivations can be seen in material objects, like calendars. He explains how the calendar we use today is based on religious ideas (like the number of the year starting from the birth of Christ and the seven day week being related to the day of rest). After the French Revolution, the Republican leaders wanted to create a calendar that would reflect their egalitarian ideas about society. 
	\item Zaruvabel uses this example of switching calendars to show that the way people count time is in many ways arbitrary. Clocks and calendars are not natural or neutral, they are social creations that reflect politics, ideology and materialized, social time.
	\item The other thing to think about is how calendars are, again like Latour says, immutable mobiles that allow for control from a distance. Calendars regular the way people spend their time, so they have an influence on human behavior even if no one is around to do so personally. 
\end{itemize}

\newpage
\section{W10: Subjective Vision}
\subsection*{Weekly Readings}
\begin{itemize}
	\item E. H. Gombrich, “The Image in the Clouds,” in \textit{Art and Illusion}, pp. 181-191,
	sections I and II.
	\item Jonathan Crary, \textit{Techniques of the Observer: On Vision and Modernity in the
	Nineteenth Century} (Cambridge, MA: MIT Press, 1992), Chapter 3, pp. 67-85.
	\item Michael Baxandall, \textit{Painting and Experience in Fifteenth Century Italy: A Primer in
	the Social History of Pictorial Style}, 2nd Edition (Oxford: Oxford University Press,
	1988), pp. 29-36 (Sections 1-3).
\end{itemize}

\subsection{Notes on Readings}
\subsubsection{The Image in the Clouds}
Apollonius, a Pythagorean sage chatting with Damis discussing concepts
\paragraph{What's a painting?}
\begin{itemize}
	\item \textbf{Mimesis}: Replicating Reality
	\begin{itemize}
		\item Is there such a thing as a painting? Why mix colors? $\rightarrow$ For the sake of imitation.
		\item Are the things we see in the sky and in reality works of imitation too? Is a God a painter?
		\item Clouds only have meaning by chance.  It is us who by nature are prone to imitation and articulate these clouds
		\item Imitation takes the effort of both the creator and the beholder $\rightarrow$ This skill is called \textit{imitative faculty}. 
		Under the name of \textit{projection}, this faculty has become the focus of interest for a branch of psychology e.g. Rorschach test for
		\textit{Paredolia}, the phenomenon that occurs when we see faces and body parts in inanimate objects or images 
	\end{itemize}
\begin{figure}[H]
	\centering
	\includegraphics[width=0.2\linewidth]{images/Rorschach_test.png}
\end{figure}
	\item \textbf{Schemata}: Accidental forms of projection (psychologist F.C. Ayer)
	\begin{itemize}[label=$\circ$]
		\item The method of projection: Creating a support for the artist’s memory images. 
		\item It does not matter whether the initial form that the artist try to project is man-made or found, what matters is what he can make of it. 
		\item Alexander Cozen wrote a strange book \textit{A New Method of Assisting the Invention in Drawing Original Compositions of Landscape},
		which involves the use of accidental inkblots for the suggestion of landscape motif for amateurs - it was considered a deliberate challenge to traditional ways of teaching art. 
		\begin{itemize}[label=\tiny$\blacksquare$]
			\item \textit{“... too much time is spent in copying the work of others, which tend to weaken the powers of invention”}
			\item \textit{“I lamented the want of a mechanical method sufficiently expeditious… to draw forth ideas of an ingenious mind disposed to the art of designing.”}
			\item \textit{“To sketch is to delineate ideas; blotting suggest them.”}
		\end{itemize}
	\end{itemize}
	\item Projection is formed from new combinations and variations of ideas that the beholder desires, rather than being ingrained an entirely new vocabulary
	\item \textbf{Paganism: polytheism}; a belief in false god(s)
	\item The power of \textit{``confused shapes'' (Leonardo da Vinci)} $\rightarrow$ don’t do meticulous drawing. Rapid and untidy sketches make it easier to generate possibilities
	\item Van Goyen uses his own unfinished work as a base for expanding ideas 
\end{itemize}


\paragraph{Question Prompts from Samson}
\begin{itemize}
	\item What is the author telling us about painting? 
	\item What passages in the text do you get your interpretation from?
\end{itemize}

\subsection{Techniques of the Observer: On Vision and Modernity in the 19th Century}
\begin{enumerate}
	\item Can you identify what ‘argument’ Crary is trying to make about how people conceptualize the act of seeing?
	\item What passages in the text do you base your interpretation from?
\end{enumerate}

\subsubsection{Subjective vision and the separation of the senses}
\begin{itemize}
	\item \textbf{Theory of Colours} (German: \textbf{Zur Farbenlehre}) is a book by Johann Wolfgang von Goethe about the poet's views on the nature of colours and how these are perceived by humans. It was published in German in 1810 and in English in 1840.
	\item Look at bright shit in a dark room and when the light goes off you see this weird outline thing where the light used to be. It's not some property of light, but rather a consequence of how the eyes capture and process light to form images in your mind
	\item The body generates these extra coloured bits: it has become an active producer of the optical experience
	\item What is important about Goethe's account of subjective vision is the inseparability of two models usually presented as distinct and irreconcilable: a physiological observer who will be described in increasing detail by the empirical sciences in the nineteenth century, and an observer posited by various ``romanticisms'' and early modernisms as the active, autonomous producer of his or her own visual experience
	\begin{itemize}[label=$\circ$]
		\item What this means is that even “empirical” observations are the result of people producing their own visual experience, in which case it no longer is empirical
	\end{itemize}
	\item Kant says that when we see shit, we don't see the thing as it appears in the world, we see the thing as what our senses tell us they look like
	\item The result of this realisation is that \textbf{vision is no longer the means by which we observe, but a subject of observation} as well
	\item Instead of looking only at optics from POV of how light behaves, we also examine the eyes: how they capture light and transmit the info to us so we can see shit
	\item The camera obscura is an idealised form of how vision works: the human body is scrub and thus we get the sparkly crap when we close our eyes. While in the past it is thought that internal representation and external object can be divorced, now they are thought to be inseparable
	\item Instead of looking at stuff empirically and objectively, we should also look at how what we see as objective knowledge gained from the senses and how our senses interact. 
	\item \textbf{Physiology is an important part of how we perceive and therefore observe things}
\end{itemize}

\subsubsection{Painting and Experience in 15th Century Italy}
\begin{enumerate}
	\item What do you think he means by the term ``cognitive style''?
	\item What influences the act of `seeing'? Where in the text does he talk about this?
\end{enumerate}

\paragraph{The period eye}
\begin{itemize}
	\item While the physical means by which visual information is transmitted to the brain is similar in most humans, the way in which our brains process and thus how we perceive is different
	\item We use machine learning from the dataset of our experiences to make sense of visual data. How is the object coloured? What is its texture? We obtain judgments by referencing past experiences. But this way of making sense of the world comes at the cost of ambiguity and subjectivity
	\item How we interpret an object depends on many things, such as context and personal skill, the patterns we recognise, how we tend to infer and draw analogies: AKA our \textbf{COGNITIVE STYLE}
	\item Dude uses Euclid's geometry thing to illustrate that the rectangle looking thing will be more likely to appear to be a rectangle with a circle overlain as opposed to a circular thing with L shapes protruding out IF they had read Euclid’s geometry field guide
	\item If you add context, the possible interpretations change again: the aforementioned rectangular object has a caption that tells people that it represents something, which makes people search through their past experiences to see if they can find a relationship and therefore a representation of something they recognise
	\item If we say that the thing was a floorplan, an Italian architect might infer a circular building with rectangular halls while a Chinese one might infer a circular central court in a rectangular temple
	\item We use context and experience to compensate for \textit{incomplete information} from sense-perception about objects and the facets of the process of \textbf{this compensation occurs in parallel, not sequentially}
	\item When we look at a drawing, we are abstracting the marks on the paper as representations of the actual thing irl. There are caricatures, which are more direct forms of abstraction and there are conventional representations of objects which may not resemble the actual objects but serve adequately as representations given sufficient contextual knowledge: one is a drawing of a mermaid, which sort of looks like a mermaid. The other is a bunch of squiggly lines that don't really look like water but since they are surrounding the mermaid so you can infer that it represents water.
	\item Cognitive style also affects the way we look at paintings. For example, if a guy with knowledge on perspective looks at a painting, surely he would obtain a different experience than one without
	\item \textbf{Personal taste} can be considered the \underline{conformity between the technical skill demanded by the}\\\underline{painting and the actual skill the viewer possesses}: some normal scrub wouldn't appreciate the painting as much as someone who is a \textit{painting sommelier}
	\item The experience of viewing the painting might also differ with \textbf{background}. An Italian guy might recognise a painting of a building in Italy correctly as an indoor church thing while a Chinese guy seeing the same painting might recognise an open-air courtyard. Assumptions of what belongs where and what looks like what can skew interpretations
	\item ●	This ambiguity is not necessarily a lack of skill on the part of the painter: he might be relying on the presence of \textbf{contextual knowledge} to convey meaning and information
\end{itemize}

\subsection{Notes on Class Recitals}
\begin{itemize}
	\item 17th century painters and mimesis: if the painter was good enough, then the picture can speak for itself. You see the pic and you can immediately see what it is supposed to represent
	\item Some old age: use personal judgement to determine if something is representative of something
	\item Mechanical objectivity in science: take pic and let the viewer interpret
	\item Gombrich: the experience of looking at a painting differs from person to person
	\item Quattracento man's culture
	\begin{itemize}[label=$\circ$]
		\item Painters care more about the opinions of people who are able to appreciate them
		\item The patronising class
		\item You can talk about social class based on how people see paintings 
	\end{itemize}
	\item Forms and styles of painting respond to social circumstances
	\item But you can also see social circumstances from forms and styles of painting from that time and place
\end{itemize}

\subsubsection{Prof Samson's thoughts}
Recall that this class asks you to think about images in terms of production, networks, and ‘seeing’ or ‘reception’. The first 2/3rds of the course were mostly about production (e.g. rendering, transcriptions, standards, mapping, etc.) and networks (e.g. Latour's immutable mobiles and networks of inscriptions). We are now focusing more on the `seeing' aspect of the image. So, just as a lot of things can affect the production of an image (e.g. human judgment, economics, politics, etc.), the same can be said of `seeing'. Because people see and people have different ideas, experiences, and beliefs, sometimes the meaning each person attaches to an image will vary. From this, we can begin to see why the idea of `visual culture' is important. It allows us to think about ways of seeing as a social phenomenon, not just a biological one.

\medskip

\noindent\textbf{Representation, Mimicry, Accuracy}\\
In scientific presentation, and in social sciences, there is an underlying notion of mimesis: images reflect something out there in the world. But as the readings have shown, whether territory of lizard, rat brain tissue, national territory, time, there is a line of thinking that says that images produce (not merely reflect) in a hybrid way the object they represent. So the mimetic model is not always the right one to use for understanding images.

\medskip

\noindent\textbf{Gombrich}

\medskip
\noindent\underline{Coding, p. 181}

\medskip
\noindent Coding means here lots of things inform the way someone sees an image: economics, religion, political views, etc. This we've seen over and over in previous readings with regards to the production of images.

\medskip
\noindent Coding in the production of images implies there is an artist or someone that makes an image with an intention to communicate a meaning or likeness of some object (like a horse). The artist or scientist is coding things into the image as they make it. They follow conventions and other ideas about what should be depicted and how.

\medskip
\noindent\underline{Clouds and Intention, p. 182}

\medskip
\noindent Because people see images in things like clouds and the stars, etc., it is clear that the beholder of an image has ideas or the `mimetic faculty' just like the artist or the scientist. People see things in nature, where there is no intention or `coding'. The beholder, then, is the person doing the `work' with images.

\medskip
\noindent\underline{General Conclusions}

\medskip
\noindent Images are interplay between people and thing, they are `hybrid objects'. Their meaning is not fixed. 

\bigskip

\noindent\textbf{Crary - Two Models of Vision}

\medskip
\noindent\underline{Camera Obscura, p. 68}

\medskip
\noindent If we think about images using the camera obscura as a model of vision, we may think of it as an external, mechanical, and natural process. Light comes in through an opening/lens, and it is focused and makes an image on a surface. There is no human intervention in this process. In the model of the Camera Obscura, the human body is not part of the equation.

\medskip
\noindent\underline{Physiology and Vision, p. 68}

\medskip
\noindent Yet, if we close our eyes after looking at a light source, what happens? The mind still can see colors and shapes. Images are in the mind as well as `out there' in the world. As such, vision is subject.

\medskip
\noindent What an image is, or what it means, then is the result of the interaction between human body and external stimuli. It is neither only material or only physiological

\bigskip
\noindent\textbf{Baxandall}

\medskip
\noindent So if the images are partly in the eyes/mind, how do we see the same things? 

\medskip
\noindent\underline{Cogntive Style}

\medskip
\noindent A shared understanding of shapes, colors, etc. that equate to standardized meanings.

\medskip
\noindent\underline{Visual Cultures}

\medskip
\noindent Similar readings/interpretations - the transformation of raw data into meaning in predictable, relatively stable ways - happen when people share a common `visual culture' or ways of seeing. That is, if seeing is done by observers, observer must have common understanding of how to `read' image. Factors in reading images include:
\begin{itemize}
	\item Common conventions of representation
	\item Common experience of object (say a building)
	\item Common assumptions about things being represented and about social life
\end{itemize}

\newpage
\section{W11: Non-human Vision}
\subsection*{Weekly Readings}
\begin{itemize}
	\item Paul Virilio, \textit{The Vision Machine} (Bloomington: Indiana University Press, 1994), pp.
	59-62.
	\item Catelijne Coopmans, “Visual Analytics as Artful Revelation,” in \textit{Representation in
	Scientific Practice Revisited} (2014), pp. 37-59.
\end{itemize}

\subsection{Notes on Readings}
\subsubsection{Visual Analytics as Artful Revelation}
\begin{itemize}
	\item Economist special report: one of them talks about data visualisation. Humans are ``naturally'' predisposed to looking at patterns and so while they suck at looking at raw data, they are good at interpreting graphsMaking visualisations are now facilitated by machines and shit. Comsci, statisticians, artists, storytellers work with and extract data from big dataset
	\item Visual analytics very useful mmmm
	\item Free visualisation tools and open source stuff, users can share and create
	\item Like the visualisation techniques of the past, visual analytics reveals what was hidden before
	\item Recall the previous weeks where the task of interpretation shifts from the content creator then to the viewer. Now this responsibility is co-shared
	\item With computers, viewers can manipulate and uncover extra insights not intended by the creators of the graphi
	\item \textit{Revealing stuff} both enables and is enabled by visual analytics as a way of working with data 
	\item Data as a source of value, and visualisation as a means of unlocking this value
\end{itemize}

\paragraph{Artful relevation}
\begin{itemize}
	\item When you say representation is \textit{artful}, you imply the inefficacy of scientific tools
	\item \textit{Image processing} brings together the act of making stuff visible and the act of interpreting this stuff 
	\item Rendering image data to highlight the observable and measurable properties of data can be considered \textit{artful}
	\begin{itemize}[label=$\circ$]
		\item But if it is \textit{artful}, then it might also be considered unscientific
		\item Scientists might consider some visualisations unscientific, or to denounce the production of ``pretty pictures'' as unscientific
	\end{itemize}
	\item Visual analytics, while software supported, is not automatic and users have to make judgements
	\begin{itemize}[label=$\circ$]
		\item Visual analytics also encourages neutrality in what a display is supposed to show: ``you might not know what you are looking for, but once you see it, you'll know it''
		\item Instead of presupposing something within the data, it presupposes ability on the part of the user
		\item ``A visualisation is not a representation but a means to a representation''
		\item The data is there in ``plain sight'' for viewers to make sense of. Instead of telling what the user to see, visual analytics empowers people to see
	\end{itemize}
\end{itemize}

\paragraph{Watching webinars}
\begin{itemize}
	\item Webinars: promote visual analytics as a way of seeing-into-data
	\begin{itemize}[label=$\circ$]
		\item A form of subtle marketing
		\item Some things were visually made to stand out
	\end{itemize}
	\item Developing insight into data
	\begin{itemize}[label=$\circ$]
		\item The emphasis on surface views as beacons and shortcuts
		\item Excel support charting but not true visual analysis
		\item Tableau can display a series of miniature graphical displays that only differ on some parameters, so that their combination support analysis through immediate visual comparison
		\begin{figure}[H]
			\centering
			\includegraphics[width=0.3\linewidth]{images/data}
		\end{figure}
		\item Convenient “Show Me” Button that generate charts for parts of the data that they have selected. They can click it again to display a different view, so that they dun have to ``code'' it as manually
	\end{itemize}
	\item Visual Analytics tools are ubiquitous online and available free.
	\item Data Scientist is the sexiest job of 21st century - Davenport and Patil 2012
	\item However, there are usually disclaimers in webinars saying that they are visualizing fake/altered data for convenience/privacy sake
	\begin{itemize}[label=$\circ$]
		\item These form of announcements that deny the reality of what is seen serves to complicate the notion that the webinars can help prospective users truly understand what it takes to get value from visual analytics.
		\item This chapter talks about v common sense stuff 
	\end{itemize}
	\item Rhetoric of visual analytics: Concurrence of artfulness and revelation
	\item \textit{Immanence} can be expressed through visualization of the significant correlations in the data
	\item Audience can follow how a dataset is made to give up its secrets as the presenters make things pop, oscillate between complexity and simplicity
	\item The promise of sight as the key to knowledge
	\item Stereotypes of visual analytics
\end{itemize}

\subsubsection{The Vision Machine}
\begin{itemize}
	\item \textit{``Now objects perceive me''}
	\item Sightless Vision: Video camera controlled by computer
	\begin{itemize}[label=$\circ$]
		\item Automation of vision
		\item Splitting of viewpoint: sharing of perceptron of environment between animate and inanimate
		\item Instrumental virtual images == foreigner's mental pictures
	\end{itemize}
	\item Philosophical and scientific debate shifted the question of objectivity of mental images to the question of their reality.
	\begin{itemize}[label=$\circ$]
		\item Images are \textit{mental objects - J.P. Changeus}
		\item The invention of photography/biography forces us to have a preconceived idea of how to perceive and mentally retain images
		\item We somehow can accept that the image produced by the camera is true and not the ``\textit{virtual}'' reality that is resulted from the virtual image
		\item High speed photography =  videography = cinema possible = paradoxical real nature of virtual imagery
		\item Any split second of exposure time involve some form of memorisation by the user according to the speed of exposure. So if a film is projected at >60 fps, we feel like it is very fluid.
		\item \textit{Celluloid support surface: Exposure Time allows/edits the act of seeing.}
	\end{itemize}
	\item If seeings is in fact foreseeing
	\begin{itemize}[label=$\circ$]
		\item Forecasting industries being possible:  where vision machines are designed to see and foresee in our place
		\item Synthetic-perception machines can replace us not because our ocular system’s limited depth of focus but more because of limited depth of time of our physiological perception. 
	\end{itemize}
	\item Kinematic Energy: Energy resulting from effect of movement and its varying speed on ocular/optical/optoelectronic perception
	\item Space of Sight = Minkovskian event-space, relative-space
	\begin{itemize}[label=$\circ$]
		\item We can see because light is \textit{weak}
		\item We understand our present environment from our distant visual memories. 
		\item Looking is then the act of recording these distant visual memories
	\end{itemize}
\end{itemize}

\subsubsection{John Berger - Ways of Seeing episode 1}
\begin{itemize}
	\item Seeing is not natural: it is dependent on convention/habit. Culture!
	\item An image might lose meaning if moved out of its `original' context
\end{itemize}

\subsection{Notes on Class Recitals}
\begin{itemize}
	\item We bear cultural burdens when we look at images
	\item Geertz! - human behaviour influenced by culture
	\item But how do we get rid of this subjectivity shit? We use machines.
	\item Coopman says that the meaning we derive from a dataset might be merely a part of how we are conditioned to think through our culture and experiences.
	\begin{itemize}[label=$\circ$]
		\item The data might be immanent, but the conclusions might not
		\item The people that look at and interpret data must have some skill or context to do so successfully
		\item The data does not inherently contain the juicy deets
		\item But visual analytics is promoted as a software that allows any random scrub to analyse data, with open source and shit
		\item But you actually really need human interaction and expertise to interpret data
		\item ``Artful' implies human skill in data analytics
		\begin{itemize}[label=\tiny$\blacksquare$]
			\item Maybe the art comes from how the interpretations are presented?
			\item As opposed to the actual interpretation
		\end{itemize}
		\item But people see data analytics as factual
	\end{itemize}
	\item Virilio:
	\begin{itemize}[label=$\circ$]
		\item ○	Object $\rightarrow$ camera $\rightarrow$ machine
		\item While images used to be windows to the past, we are now using images to predict
		\item He says the shift to a more predictive role is bad. Leave the analysis to the humans! 
		\item interpretability in computer vision
		\item Industrialisation of vision and Automation of perception: the need for fast af response, when human perception and response is too slow
	\end{itemize}
	\item There has been an obsession with removing humans from the equation: to go beyond human limits and the quest for objectivity
	\item Now we have machines to do the seeing for us
	\item Humans use context to interpret images. How do you feed context to the machines?
	\item Context is not immutable mobile. 
	\item But since you are using context you end up making machines subjective again because the decisions made in choosing what context to include is in itself subjective
	\item Input in human and computer vision are quite different
\end{itemize}

\subsubsection{Video: Ways of Seeing}
The video was simply a way to summarise some of the things we've been talking about during the course of the semester. The topic of `seeing' is prominent in John Berger's narration. He believes that `seeing' (here meaning the assignment of meaning to images) is not simply `natural'. Instead, it is learned. This is in line with the arguments we covered last week about `visual culture'. That is, the meaning of an image depends in part on things like conventions and experience and context. It is not simply a mechanical activity.

\medskip
\noindent The other theme that came up in the video had to do with networks and reproduction of images. Once people can reproduce an image, that image can spread widely and in different contexts that it was initially intended for. As such, it's meaning and uniqueness (or aura) can change, at least according to Berger.

\subsubsection{Coopmans}
A few questions to think about for Coopmans’ article.

\paragraph{Art and Data Visualisation}
\begin{itemize}
	\item The title of the article is ``Artful Revelation''. Think about what this might mean, why she chooses this phrase. What is `artful' about the representation of data in the supposedly objective disciplines of social sciences?
\end{itemize}

\paragraph{Agency/Observer}\mbox{}

\medskip
\noindent Where do the `insights' from looking at visualizations of data come from? The viewer? The data? The image? Real phenomena?

\medskip
\noindent Is meaningful information contained in the data sets themselves or are they produced in the process of creating a visualization of the data? Or does `insight' more complex? That is, does the information only become meaningful when a human observer looks at the visualization? 

\medskip
\noindent Who or what in visual analytics is doing the `artful revelation' of meaningful data? Can or should machines do the subjective interpretive work typically performed by humans?

\paragraph{Patterns/Ideals/Referents}\mbox{}

\medskip
\noindent Think about this sentence on p. 40: ``Instead of presupposing a referent [note: a signified thing out there in the world], visual analytics presupposes users' pattern recognition ability...'' What does it imply about how data analytics works?

\medskip
\noindent Things like `opportunity' or `insights' are not material objects. So how do we know we are seeing them when we look at visual images of data? 

\medskip
\noindent\textbf{Immanence} here is simply the idea that meaning is inherent in the data and all we have to do is look at it. It reveals itself (with some work using visualization tools) because it is `in' the data.

\medskip
\noindent\textbf{Conditionality} here is when meaning (of data sets) is thought to be dependent on something else, like a way of representing information or a missing set of numbers or the skill of the person looking at data visualizations.

\subsubsection{Virilio}
The section of Virilio's essay that we read does not really present an `argument' as much as it is an exploratory piece about the development of `seeing' in non-human (machine) things. A self-driving car, for example, can be thought of as `seeing' the road. But what does it see? The same thing as a human or something else?

\medskip
\paragraph{Virtual Image}\mbox{}

\medskip
\noindent Non-material image, referring here to mostly the pictures one `sees' in one's mind.

\medskip
\paragraph{Material Image}\mbox{}

\medskip
\noindent A tangible thing, like a photograph.

\medskip
\paragraph{Who Sees?}\mbox{}

\medskip
\noindent The responsibility for finding meaning in images has typically been the human observer. But now the task of `looking' can be said to have been split between humans and machines. He feels that we cannot predict what a machine will see, or more precisely how it will interpret the information that it receives. 
\end{document}